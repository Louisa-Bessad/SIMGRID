Dans cette section, nous avons pu voir qu'il existe différents types de virtualisation. Puisque nous souhaitons pouvoir utiliser des milliers de plateformes durant nos exécutions nous ne pouvons utiliser la virtualisation complète de la machine. La virtualisation légère par dégradation est également exclue car nous devons pouvoir émuler des machines plus performantes que l'hôte. La virtualisation légère par interception semble être celle qui correspond le mieux aux besoins de notre projet. Pour pouvoir être mise en place elle nécessite d'utiliser des outils d'interception pour les différentes actions de l'application. Il existe différents approches permettant de faire cela, nous allons les présenter dans la section suivante.
