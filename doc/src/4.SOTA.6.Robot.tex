\subsection{Robot}
\label{subsection:Robot}

Robot \citep{ROBOT_web} est une plateforme de tests automatisée pilotée par mots-clés\footnote{Utilisation de mot-clés dans la définition des données de test pour déterminer les actions à prendre sur les données lors de l'exécution. Voir keyword-driven testing  \url{https://en.wikipedia.org/wiki/Keyword-driven\_testing}} créée en 2005 par Pekka Klärck \citep{ROBOTlaukkanen2006data}. La plateforme est libre de droit et la majeure partie de ses outils et librairies le sont égalements. Elle permet de lancer des tests ``normaux'' ou du ``ATDD''\footnote{Acceptance Test-Driven Development \citep{ROBOTlarman2010practices}: Techniques consistant à utiliser des exemples/tests pour clarifier et documenter les exigences d'un test. Également appelée ``Agile Acceptance Testing'' ou ``Story test-driven development''}

Robot possède une architecture modulaire extensible
Figure \ref{Robot_arch}.  Pour créer un test, on le décrit sous forme de
tables et on créé une librairie associée qui les transformera en un
exécutable lorsqu'elle sera appelée. Il existe quatre types de tables
\textit{i)} la table de paramétrage pour importer des fichiers ou
définir des métadonnées, \textit{ii)} la table des variables pour
déclarer des variables globales pouvant être utilisées ailleurs,
\textit{iii)} la table des \textit{cas de test} contenant les cas
réels à tester, \textit{iv)} la table des \textit{mots-clés} qui
contient les mots-clés utilisateurs utilisés pour construire les
\textit{cas de test}.  Les mots clés sont de deux types dans Robot;
les mots-clés utilisateurs définis dans les tables de mots clés et les
mots clés de librairies implémentés par du code dans les librairies de
test.

\begin{figure}
  \centering
  \includegraphics{Pictures/png/Robot_architecture}
  \caption{Schéma de l'architecture de la plateforme Robot}
  \label{Robot_arch}
\end{figure}

Au lancement d'une exécution la plateforme, Robot va analyser les tables décrivant le test en fonction des mots clés utilisés. Les bibliothèques fournissent les ``mots-clés de librairie''\footnote{L'architecture étant modulaire si une bibliothèque n'existe pas on peut la créer et l'ajouter facilement a une des trois sections de bibliothèques existants: standard (os, screendshot), externes (ssh, http, ftp) et spécifique à des projets (Android, iOS, Eclipse). On créé ainsi de nouveaux ``mots-clés de librairie''.} que va utiliser Robot pour intéragir avec le système à tester. Les bibliothèques peuvent communiquer directement avec le système ou utiliser d'autres outils de tests (par exemple les drivers). Les outils sont là pour faciliter la création et l'exécution des tests: ``libdoc'' pour générer les ``mots clés de librairie'' et fichiers sources, plug-in vim pour le développement, emacs major mode pour editer les données de test... 

La plateforme dispose également d'un système de tag permettant d'ajouter des métadonnées catégorisant les tests, de collecter automatiquement les statistiques d'un test, selectionner les tests à exécuter et spécifier les tests qui sont critiques. 

L'exécution rend deux fichiers le ``report'' et le ``log'' décrivant ce qui s'est passé durant l'exécution du test. Malgré le détail de ces deux fichiers on ne peut pas suivre l'exécution en temps réel du test.

