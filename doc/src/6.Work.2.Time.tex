\subsection{Temps}
\label{section:work:time}

Comme détaillé en section \ref{subsubsection:time}, une médiation des fonctions ayant trait au temps est nécessaire pour virtualiser l'environnement. En effet, nous ne pouvons pas laisser les applications accéder aux horloges de la machine hôte, utilisant pour cela la bibliothèque \texttt{VDSO}. Nous avons alors proposé de créer une bibliothèque ré-implémentant les fonctions temporelles et d'utiliser la variable d'environnement \texttt{LD\_PRELOAD}, présentée en section \ref{paragraphe:LDPreload}, pour exécuter les fonctions de notre bibliothèque. De cette façon, on bloque les appels à la bibliothèque \texttt{VDSO} et l'application ne peut plus accéder au temps de la machine hôte. Néanmoins, pour que notre virtualisation soit parfaite il faudrait que le temps que l'application voit s'écouler soit celui qui s'écoulerait sur la machine simulée par SimGrid. 

Pour permettre cela, nous avons créé une bibliothèque de fonctions temporelles que nous plaçons dans la variable d'environnement \texttt{LD\_PRELOAD} à chaque exécution d'une application avec Simterpose. Cette bibliothèque contient pour chaque fonction de la \texttt{libc} faisant appel à une des horloges de la machine (\texttt{ftime}, \texttt{time}...) une nouvelle fonction de même prototype. Au lieu de demander l'horloge de la machine hôte, les nouvelles fonctions de temps demandent à SimGrid de fournir l'heure sur la machine qu'il est en train de simuler. Comme le montre la figure \ref{time_interception}, l'appel à la fonction de temps se fait dans l'application qui s'exécute sur le thread espionné de Simterpose. Ce dernier est contrôlé par le processus espion via l'intercepteur \texttt{ptrace} qui est le seul processus de Simterpose à pouvoir communiquer avec le simulateur. Nous devons donc passer par le thread espion pour pouvoir interroger SimGrid. 

Cependant, le seul moyen de faire intervenir le processus espion est d'effectuer un appel système dans l'application en train de s'exécuter. Dans notre cas, c'est la nouvelle fonction temporelle qui doit effecteur l'appel système. En effet, le processus espion va intercepter l'appel et exécuter le "handler" qu'on lui aura fourni, comme pour le réseau de communications, avant de rendre la main à l'application en plaçant dans le registre de retour de l'appel système l'horloge que lui aura donné le simulateur.

Néanmoins, la fonction temporelle ne doit pas utiliser n'importe quel appel système pour récupérer l'heure simulée. Par exemple, si on choisit de modifier l'appel système \texttt{send}, le handler de ce dernier contiendra l'appel à la fonction \texttt{MSG\_get\_clock} puis le neutralise. On ne pourra donc plus utiliser cet appel pour échanger sur le réseau. Dans tous les noyaux, il existe des appels systèmes qui ne sont plus implémentés; c'est le cas de l'appel système \texttt{tuxcall}. Quand on fait appel à ce dernier le système ne fait rien, il émet juste un avertissement pour dire que l'appel n'existe plus. De plus, comme cet appel ne produit aucun action de la part du système il n'est pas nécessaire de le neutraliser. Nous avons donc choisi de placer notre handler sur cet appel système.

Les Algorithmes \ref{ALGO_TIME} et \ref{ALGO_TUXCALL} ainsi que la Figure \ref{time_interception} nous montrent respectivement l'algorithme de la nouvelle fonction \texttt{time}, le handler de l'appel système \texttt{tuxcall}, et ce qui se passe dans SimGrid et Simterpose quand on veut récupérer l'heure simulée grâce à la double interception complémentaire de \texttt{LD\_PRELOAD} et 2.

\begin{figure}[H]
  \centering
  \includegraphics[scale=0.7]{Pictures/png/Open_pandor}
  \caption[Déroulement de l'appel à une fonction temporelle réimplémentée]{Déroulement de l'appel à une fonction temporelle réimplémentée:}
  \label{time_interception}
\end{figure}

\begin{center}
\begin{minipage}{10cm}
\begin{enumerate}
\item Interception de l'appel à la fonction \texttt{time}
\item Interception de l'entrée à l'appel \texttt{tuxcall}
\item Execution du handler d'entrée de \texttt{tuxcall}
\item Rend la main à l'application pour exécuter l'appel système
\item Interception de la sortie de l'appel \texttt{tuxcall}
\item Execution du handler de sortie de \texttt{tuxcall}
\item Appel à \texttt{MSG\_get\_clock}
\item Rend la main à l'application
\item Appel à \texttt{time} terminé
\end{enumerate}
\end{minipage}
\end{center}

\newpage
\lstdefinestyle{customc}{
  belowcaptionskip=1\baselineskip,
  breaklines=true,
   frame=L,
  %%frame=single,
  xleftmargin=\parindent,
  language=C,
  showstringspaces=false,
  basicstyle=\footnotesize\ttfamily,
  keywordstyle=\bfseries\color{blue},
  commentstyle=\itshape\color{green!40!black},
  %%identifierstyle=\color{blue},
  stringstyle=\color{violet},
}
\lstset{escapechar=@,style=customc, caption=[Fontion \texttt{time} et appel à \texttt{tuxcall} côté processus espionné{\color{white}blablablablablablablabla}]{Fontion \texttt{time} et appel à \texttt{tuxcall} côté processus espionné}, label=ALGO_TIME, captionpos=b}
\begin{lstlisting}
  /* Macro to ask the clock to SimGrid */
  #define get_simulation_time(clock) \
  syscall(SYS_tuxcall, clock)

    /* Time function */
    time_t time(time_t *t){
      double* sec = (double *) malloc(sizeof(double));
      get_simulation_time(sec);
      if (t != NULL)
      t = (time_t *) sec;  
      return (time_t)(*sec);    
    }
\end{lstlisting}

\lstset{escapechar=@,style=customc, caption=[Handler pour l'appel système \texttt{tuxcall} côté processus espion{\color{white}blablablablablablabla}]{Handler pour l'appel système \texttt{tuxcall} côté processus espion}, label=ALGO_TUXCALL, captionpos=b}
\begin{lstlisting}
  void syscall_tuxcall(reg_s * reg, process_descriptor_t * proc){

  if (proc_entering(proc))
    proc_inside(proc); /* syscall_tuxcall_enter */
  else
    syscall_tuxcall_post(reg, proc);  /* syscall_tuxcall_exit */
  
}

void syscall_tuxcall_post(reg_s * reg, process_descriptor_t * proc){

  proc_outside(proc);
  XBT_DEBUG("tuxcall_post");

  /* Ask the clock to SimGrid */
  double clock = MSG_get_clock();
  
  /* Put the return value in argument register of the syscall */
  /* Uses ptrace because the call is in the memory area of another process */
  /* You cannot access register directly */
  ptrace_poke(proc->pid, (void *) reg->arg[1], &clock, sizeof(double)); 

  if (strace_option)
    print_tuxcall_syscall(reg, proc);
}
\end{lstlisting}

