\section{Conclusion}
\label{section:ccl}

L'objectif de ce stage était de montrer qu'il est possible de faire de la
virtualisation légère afin de tester des applications distribuées large échelle
quelconques. Pour cela, nous avons commencé par expliquer le concept de
virtualisation légère et pouruqoi nous l'avions choisi. Puis nous avons présenté
les outils permettant sa mise en place ainsi que les projets qui actuellement
font de la virtualisation légère pour différents types d'applications. Les
projets existants ne permettant pas de résoudre les quatre problèmes engendrés
par cette virtualisation (gestion du temps, des threads, des communications
réseaux et le DNS) et donc de tester n'importe quel type d'applications
distribués, un nouveau projet a été lancé.

L'émulateur Simterpose dévelopé au LORIA permet d'exécuter et de tester des
applications distribuées réelles, sans disposer de leur code source, dans un
environnement virtuel. Il se base sur la plateforme de simulation SimGrid pour
mettre en place l'environnement d'exécution dans lequel l'application pensera
s'exécuter. Pour maintenir la virtualisation, les actions des applications sont
interceptées et modifiées pour ensuite être exécutées. On utilise SimGrid pour
calculer la réponse de l'environnement virtuel aux différentes actions. La
solution proposée intercepte les actions à deux niveaux différents: appels
systèmes et bibliothèques. Simterpose permet également d'injecter diverses
fautes dans la simulation pour avoir une virtualisation plus réaliste.

Au début du stage, Simterpose gèrait plus ou moins bien les threads, le réseau
de communications n'était implémenté qu'en parti et les deux autres
fonctionnalités (temps et DNS) étaient inexistantes. Nous avons donc terminé
l'implémentation du réseau de communications et avons mis en place la gestion du
temps. Ces deux fonctionnalités ont nécessitées la création de nouveaux outils
et des améliorations ont été apportées à Simterpose. Cependant, comme nous avons
pu le voir notre émulateur n'est pas encore terminé. Néanmoins, nos expériences
nous ont permis de montrer qu'il est déjà possible de mettre en place une
virtualisation légère par interception pour tester des applications distribuées
quelconques pour les deux fonctionnalités que nous avons implémentées.
