\section{Approches possibles pour la virtualisation légère}
\label{section:emulation}

Dans cette section, nous présentons les différentes approches possibles pour
virtualiser des applications distribuées afin de permettre leur
étude. L'objectif général étant de permettre l'étude du comportement d'une
application réelle sur une plateforme différente de celle sur laquelle
l'expérience a lieue. On appellera la vraie plateforme hébergeant l'expérience
``machine hôte'' et celle sur laquelle on veut tester l'application ``machine cible''.

Il existe actuellement deux méthodes permettant de faire de la virtualisation:
standard et légère.

La première est une émulation complète de la machine comme celle utilisée par
vmware. Dans cet émulateur, le système d'exploitation est hébergé sur une
plateforme matérielle totalement virtualisée. Il n'a pas la possibilité de
lancer toutes les instructions, l'émulateur déclenche un handler pour que
l'hyperviseur les exécute. Cette solution pourrait être applicable dans le cadre
notre objectif général d'émulation d'applications mais elle ne passe pas du tout
à l'échelle (plus d'une douzaine de machines).

La seconde est dite légère car elle permet de tester des applications sur une
centaine d'instances. Pour que cela soit possible elle n'émule pas complètement
la machine, notamment le CPU. De fait, elle ne peut exécuter que des
applications ayant été compilées pour le même jeu d'instructions que celui de
l'hôte.  De plus, le choix de la virtualisation pour notre projet génère quatre
problèmes: les threads, le temps, les communications réseaux et les échange
utilisant le protocole DNS, que notre plateforme devra résoudre. Malgré cette
contrainte, c'est la virtualisation légère qui a été choisi pour notre projet
car il est très important d'avoir une solution d'exécution rapide quelque soit
le nombre d'instance.

La virtualisation légère peut se faire par limitation également appelée
dégradation ou par interception. Nous allons donc étudier ces deux possibilités
afin de choisir celle qui est la plus adaptée à notre objectif et voir comment
elle peut résoudre les problèmes cités précédemment.









