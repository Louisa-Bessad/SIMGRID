\section{Méthodes possibles pour la virtualisation légère}
\subsection{Emulation par limitation / Virtualisationi standard}
\begin{itemize}
\item principe: limiter l'accès aux ressources par exemple (cgroup, netstat, cpuburner), temps d'un SEB (bench avec netlink, limiter (cap))
\item avantage plus simple
\item désavantages: host>>target, modèle à vérifier, contrôle expérimental fin
\end{itemize}

\subsection{Emulation par interception}
 principe: interception des actions et médiation (pas juste intereption et rejeu). Intercepter des symbôles pour en changer l'effet

\subsubsection{Action sur le fichier source}
reimplem SMPI ( trop spé) ,source to source/ pass LLVM( gcc+libc=consanguin) 

\subsubsection{Action sur le binaire}
Valgrind (perf pourrie)

\subsubsection{Médiation directe des appels de fonctions}
pourquoi: pthread, temps

\paragraph{linker: LD\_PRELOAD}
pas suid
\paragraph{linker got injection}
plus dur que nécessaire

\subsubsection{Médiation des Appels Système}
pourquoi: read/write, comm reseau

\paragraph{ptrace}

\paragraph{uprobe}
Non module noyau

\paragraph{seqcomp/bpf}
Read only
