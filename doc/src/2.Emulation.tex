\section{Le choix de l'émulation}

Il existe trois façons de tester le comportement d'applications distribuées; l'exécution sur plate-forme réelle, la simulation et l'émulation. La première consiste à exécuter réellement l'application sur un parc de machines et d'étudier le comportement de l'application en temps-réel, ce que fait actuellement \textbf{Grid'5000} {\color{red}mettre citation}. Néanmoins pour mettre en \oe uvre cette solution complexe il faut disposer des architectures nécessaires pour effectuer les tests. De part le partage des différentes plateformes entre divers utilisateurs les expériences ne sont pas forcément reproductibles. 

La deuxième solution utilise un simulateurtel que \textbf{SIMGRID} {\color{red}mettre citation} par exemple. Pour on doit représenter de façon théorique l'application ainsi que l'environnements d'exécution grâce à des modèles mathématiques. Ainsi on exécute dans le simulateur le modèle de l'application dans l'environnement qui aussi modélisé. L'avantage de cette solution est qu'elle est facilement reproductible et simple à mettre en \oe uvre du moment que nous savons modéliser notre applications. Néanmoins,avec la simulation on n'exécute pas vraiment l'application. On ne peut alors valider qu'un modèle et pas l'application puisqu'on réécrit l'application selon le modèle. 

La troisième solution consiste à faire de l'émulation, c'est-à-dire que nous allons exécuter réellement l'application mais dans un environnement virtuelasés. On fera ainsi croire à l'application qu'elle s'exécute sur une machine autre que l'hôte. Il existe deux façons de faire de l'émulation: la dégradation et l'interception. Dans la première on rajoute la couche d'émulation au-dessus de la plateforme réelle (comme un hyperviseur pour une VM). Mais cela nous empêche d'émuler des machines plus puissante que l'hôte car le délai de réponse géré par l'émulateur ne peut-être inférieur à celui de l'hôte sinon l'hôte n'a pas le temps de faire les calculs nécessaires à l'application. Cette solution choisie par \textbf{Distem}{\color{red}mettre citation} est donc limitée à la capacité des plateformes à notre disposition. 
Dans le cas de l'interception, pour faire croire à l'application qu'elle s'exécute sur une machine autre que l'hôte on va utiliser deux outils; un simulateur pour virtualiser l'environnement et une {\color{red}API} qui va attraper toutes les communications de l'application avec l'hôte et qui les transmettra ensuite au simulateur. Les calculs de l'applications seront effectués sur la machine hôte mais c'est le simulateur qui calculera le temps de réponse à l'application. Pour cela il fera un rapport entre le temps d'exécution du calcul sur la machine hôte (fourni par l'API), la puissance de l'hôte et celle des machines de l'environnement que l'on simule. Le temps de l'application sera donc celui du simulateur et non le temps réel. En effet l'application quand elle fait un calcul pense être surune autre machine avec des performances différentes, elle est donc capable de savoir combien de temps prends une certain calcul sur son architecture. Hors sur l'hôte ce calcul ne prendra pas le même temps et l'application se retrouvera avec un temps prévu et un temps qui ne correspondent pas ce qui est problématique. Cette solution d'interception est implémentée dans \textbf{Simterpose}{\color{red}mettre citation}. Cette émulation peut-être faite off-line; on sauvegarde les actions de l'applications sur disque et on les rejoue plus tard dans le simulateur ou on-line; les actions sont directement reportées dans le simulateur et on bloque l'application durant le temps nécessaire calculé par le simulateur.

Dans notre cas c'est l'émulation qui a été choisie. En effet la simulation n'était pas une bonne solution pusique nous voulons valider les applications et non leur modèle, pour l'exécution sur plate-forme réelle il y avait trop de contraintes à satisfaire. Pour finir nous avons choisi l'émulation par interception car il devait être possible d'évaluer les appplications sur des machines plus puissantes que l'hôte. Nous allons utiliser SIMGRID comme simulateur et Simterpose comme API de ce simulateur. Simterpose nous permettra donc d'utiliser SIMGRID avec des applications réelles en leur faisant croire qu'elles s'exécutent sur des machines distribuées. Maintenant que nous savons comment nous allons tester les applications distribuées nous allons voir comment fonctionne notre ``émulateur'' Simterpose.
