\section{Approches possibles pour la virtualisation légère}
\label{section:emulation}

Dans cette section, nous présentons les différentes approches possibles pour virtualiser des applications distribuées afin de permettre leur étude. L'objectif général étant de permettre l'étude du comportement d'une application réelle sur une plateforme différente de celle sur laquelle l'expérience a lieu. On appellera la vraie plateforme hébergeant l'expérience ``machine hôte'' et celle sur laquelle on veut tester l'application sans y avoir accès ``machine cible''.

Il existe actuellement deux méthodes permettant de faire de la virtualisation
légère. La première est une émulation par limitation ou dégradation également
appelée virtualisation standard et la seconde est une émulation par
interception.









