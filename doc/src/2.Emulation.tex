\section{Méthodes possibles pour la virtualisation légère}
\label{section:emulation}
\subsection{Emulation par limitation / Virtualisationi standard}
\begin{itemize}
\item principe: limiter l'accès aux ressources par exemple (cgroup, netstat, cpuburner), temps d'un SEB (bench avec netlink, limiter (cap))
\item avantage plus simple
\item désavantages: host>>target, modèle à vérifier, contrôle expérimental fin
\end{itemize}

\subsection{Emulation par interception}
 principe: interception des actions et médiation (pas juste intereption et rejeu). Intercepter des symbôles pour en changer l'effet

\subsubsection{Action sur le fichier source}
reimplem SMPI ( trop spé) ,source to source/ pass LLVM( gcc+libc=consanguin) 

\subsubsection{Action sur le binaire}
Valgrind (perf pourrie)

\subsubsection{Médiation directe des appels de fonctions}
pourquoi: pthread, temps

\paragraph{linker: LD\_PRELOAD}
pas suid
On pourrait alors penser qu'une bonne solution serait d'intercepter les actions
de l'application au niveau des bibliothèques.{\color{red} rajouter des trucs sur
  VDSO}. Pour cela il existe la variable d'environnement \textbf{LD\_PRELOAD}
qui contient la liste des bibliothèques à précharger et qui est utilisée par le
noyau lors du premier lancement d'un programme. En effet par défaut Linux
effectue une édition de lien dynamiqe, l'édition de lien statique n'étant
choisie qu'en l'absence de bibliothèques partagées définissant les fonctions
utilisées par l'application. On va donc créer notre propre bibliothèque de
fonctions surchargeant chaque fonction susceptible d'être utilisée par
l'application. Une fonction surchargée contiendra alors toutes les modifications
nécessaires pour maintenir notre environnement simulé suivi de l'appel à la
fonction initiale puisqu'on souhaite juste intercepter l'appel et pas
l'empêcher. On préchargera cette bibliothèque avant les autres en la plaçant
dans la variable LD\_PRELOAD, ainsi nos fonctions passeront avant les fonctions
des bibliothèques usuelles.

Néanmoins si l'application fait un appel système directement sans passer par la
couche \textit{Bibliothèques} Fig.~\ref{AS_Communication} notre mécanisme
d'interception est countournée. En effet on ne peut surcharger que des fonctions
avec cette solution, pas des appels sytèmes. De même si on oublie de réécrire
une fonction d'une des bibliothèques utilisée par l'application. Cette solution
n'est donc pas suffisante pour le modèle d'interception que nous souhaitons
avoir.

Cependant on peut voir que LD\_PRELOAD résout les problèmes de ptrace concernant
les fonctions de temps, et inversement puisque ptrace permet d'intercepter les
appels systèmes que le modèle d'interception avec LD\_PRELOAD ne permet pas de
gérer. Une solution choisie lors d'un précédent stage est donc d'allier les
deux. On surchargera les fonctions temporelles dans notre bibliothèque
préchargée avec LD\_PRELOAD pour pallier les lacunes temporelles de ptrace. Et
pour toutes les autres fonctions ptrace s'en occupera, ainsi on est certain de
n'oublier aucune fonction. Maintenant que nous savons ce que nous devons
intercepter et comment l'intercepter nous allons voir ce que nous devons
modifier pour pouvoir maintenit notre émulation simulation.

\textit{Pour faire face à ce problème il a été choisi d'utiliser l'éditeur de
  lien dynamique \textbf{LD\_PRELOAD}, ce dernier intercepte les appels de
  l'application au niveau des bibliothèques. Pour cela on va créer une
  bibliothèque.}

\paragraph{linker got injection}
plus dur que nécessaire

\subsubsection{Médiation des Appels Système}
pourquoi: read/write, comm reseau

\paragraph{ptrace}
Nous allons intercepter les actions que sont les appels systèmes faits par
l'applciation, ainsi nous sommes sûrs d'intercepter tous les types de
communications que l'application est susceptible d'initier avec le noyau. Les
appels systèmes sont constitués de deux parties; la première, l'entrée,
initialise l'appel via les registres de l'application qui contiennent les
arguments de l'appel puis donne la main au noyau. La seconde, la sortie, inscrit
le retour de l'appel système dans le registre de retour de l'application, les
registres d'arguments contennant toujours les valeurs reçues à l'entrée de
l'appel système, et rend la main à l'application. Nous devons donc intercepter
les deux parties de l'appel système pour maintenir notre environnement simulé et
donc stopper l'application à chaque fois pour récupérer ou modifier les
informations nécéssaires avant de lui rendre la main pour entrer ou sortir de
l'appel système.

Pour faire cela il existe de nombreux outils de différents types. Il y a l'API
ptrace, qui est lui même un appel système et qui permet de tracer tous les
événement désirés d'un processus contrôlé. Néanmoins ce dernier fait de nombreux
changements de contexte pour intercepter des événements. De plus il gère mal les
processus quand on a du multithreading, et il ne fait pas parti de la norme
POSIX donc son exécution peut varier d'une machine à une autre. On trouve
ensuite les API noyau aussi appelées modules d'instrumentations telles que:
utrace, kprobes, uprobes{\color{red}citation}. Utrace fait la même chose que
ptrace mais en mode noyau. Cela permet d'éviter les nombreux changements de
contexte pour gérer l'appel système et que le noyau l'exécute {\color{green}
  clair?}. De plus il gère les événements de thread et non de processus comme
ptrace ce qui évite le problème de gestion du multhreading. Kprobes quant à lui
permet à un utilisateur d’insérer dynamiquement des points d'arrêts à des
endroits spécifiques du noyau, dans notre cas ce serait le code des appels
systèmes. Ainsi l’utilisateur peut fournir un handler particulier à exécuter
avant ou après l’instruction marquée. Quand un point d'arrêt est touché kprobe
prend la main et exécute le bon handler. Uprobes fait la même chose que kprobes
mais pour le code d'applications et pas le code du noyau. Ainsi pour chaque
point d'arrêt géré par uprobes on doit créer un module noyau qui contient le
handler à exécuter quand le point d'arrêt est atteint. Uprobes utilise utrace
pour savoir quand on atteint ou non un point d'arrêt. Les deux avantages des API
noyau est qu'elles sont rapides et qu'elles ont accès à toutes les ressources
sans aucune restriction, mais ce dernier point représente aussi leur plus gros
défaut de par sa dangerosité. De plus, dans notre cas il ne semble pas judicieux
de modifier le code du noyau ou de faire de la programmation noyau via des
modules dont il faudra également gérer le bon chargement. Malgrè ses défauts
c'est donc l'appel système ptrace qui a été choisi. Maintenant que nous savons
pourquoi nous allons utiliser ptrace comme intercepteur pour les appels systèmes
nous allons étudier son fonctionnement.

Pour intercepter les éventuels appels systèmes d'une applciation nous allons
donc utiliser l'appel système \textbf{ptrace}{\color{red}citation}. Il permet de
controler l'exécution de processus mais également d'écrire et de lire
directement dans l'espace d'adressage d'un processus. Pour cela on créé deux
processus; un qui exécutera l'application et qu'on souhaite contrôler, on
l'appellera ``processus espionné'' et un autre qui le contrôlera appelé
``processus espion''. Le processus espionné indiquera au processus espion qu'il
souhaite être contrôlé via un appel système ptrace. À la reception de cet appel
le processus espion notifiera son attachement au processus espionné via un autre
appel à ptrace. Il indiquera également sur quelles actions du processus espionné
il veut être notifié, définissant ainsi les actions bloquantes pour le processus
espionné. Dans notre cas, ce seront les appels systèmes que l'on considérera
comme point d'arrêt pour le processus espionné. Ainsi quand un des processus de
l'application voudra faire un appel système quelconque il sera bloqué avant de
l'exécuter, l'appel système ptrace sera lancé et notifiera le processus
espion. Ce dernier fera les modifications nécessaires dans les registres du
processus espionné pour conserver la virtualisation de l'environnement, puis il
rendra la main au processus espionné bloqué pour que l'appel système puisse
avoir lieu. Au retour de l'appel système le processus espionné sera de noveau
stoppé, un ptrace sera envoye au processus espion qui remodifiera les
informations nécessaires. Puis il rendra la main au processus espionné bloqué
qui sortira de son appel système avec un résultat exécuté sur la machine hôte et
un temps d'exécution et une horloge fournie par le simulateur.

{\color{red} Mettre un schema attachement attente pere attrape signal modification main fils as retour pere...}

{\color{red} \textbf{gérer cette transition}} Néanmoins il a été montré dans un
précédent stage que l'appel système ptrace est inefficace voir inutile en ce qui
concerne tous les appels systèmes temporels qu'une application voudrait
faire. \textit{les appels systèmes ``time'', ``clock\_gettime'',
  ``gettimeofday'' avec \textbf{ptrace} ne sont pas possibles, d'où l'alliance
  avec LD\_PRELOAD} Par exemple lors d'un gettimeofday l'appel système n'est pas
lancé on répond directement au niveau de la bibliothèque ainsi on n'arrive même
pas au niveau de l'appel système, donc ptrace ne fait rien.  Problème
portabilité {\color{red} \textbf{gérer cette transition}}
\paragraph{uprobe}
Non module noyau

\paragraph{seqcomp/bpf}
Read only
