\subsection{CWRAP}
%% pourquoi (tester samba), comment (LD\_PRELOAD comm, suid)

cwrap\citep{cwrap, cwrap_bis} a pour but de tester des applications réseaux s'exécutant sur des machines UNIX ayant un accès réseau limité et sans droits root. Ce projet libre a débuté en 2005 avec le test du framework ``smbtorture'' de Samba\footnote{\url{https://www.samba.org/} \\ \url{https://wiki.samba.org/index.php/Writing\_Torture\_Tests}} Pour atteindre son objectif cwrap fait de l'émulation par interception basée sur le préchargement de quatre bibliothèques via \texttt{LD\_PRELOAD}, comme nous l'avons vu en \ref{paragraphe:LDPreload}.

La première \texttt{socket\_wrapper} gère les communications réseaux. Elle modifie toutes les fonctions liées aux sockets afin que toutes les communication soient basées sur des sockets UNIX et que le routage soit fait sur le réseau local émulé. Cela permetde pouvoir lancer plusieurs instances de serveur sur la même machine hôte. On peut également utiliser les ports privilégiés (en dessous de 1024) sans avoir les droit root dans le réseau local émulé pour communiquer. Cette bibliothèque permet aussi de faire des captures de trace réseau. La seconde \texttt{nss\_wrapper} est utilisée dans le cas d'application dont les démons doivent pouvoir gérer des utilisateurs. Pour cela elle va modifier le contenu des variables d'environnement spécifiant les fichiers passwd et group qui vont être utilisés par l'application pendant la phase de test. Par défaut, les variables contiendraient les fichiers passwd et group du système mais dans ce cas le démon ne pourrait pas les modifier. \texttt{nss\_wrapper} permet également de fournir un fichier host utilisé pour la résolution de noms lors de communications entre sockets. La troisième bibliothèque appelée \texttt{uid\_wrapper} permet de simuler des droits utilisateurs. Autrement dit, elle fait croire aux applications qu'elles s'exécutent avec des droits qui ne sont pas les leurs, par exemple une exécution avec des droits root. Pour cela, on intercepte les appels de type setuid et getuid et on réécrit le mapping fait entre l'identifiant de l'appelant et celui passé en paramètre pour le remplacer par un identifiant possédant les droits désirés. La dernière libraire \texttt{resolv\_wrapper} gère les requêtes DNS. Elle intercepte ces requêtes et soit les redirige vers un serveur DNS de notre choix spécifié dans resolv.conf, soit utilise un fichier de résolution de noms que l'on a fourni à l'application.

Ainsi on a un système complet d'émulation permettant de tester des applications utilisant des réseaux complexes. Le seul bémol étant qu'on utilise uniquement \texttt{LD\_PRELOAD} pour cette émulation, il ne faut donc pas oublier une seule fonction.




