\documentclass{journal}%{IEEEtran}
\usepackage[utf8]{inputenc}
\usepackage[pdftex]{graphicx}
\usepackage[inline]{enumitem}
\usepackage{cite}
\usepackage{url}
\usepackage{color}
%\ifCLASSOPTIONcaptionsoff
%\usepackage[nomarkers]{endfloat}
%\let\MYoriglatexcaption\caption
%\renewcommand{\caption}[2][\relax]{\MYoriglatexcaption[#2]{#2}}
%\fi
\linespread{0.9}
%\graphicspath{{include/img/}}
%\DeclareGraphicsExtensions{.jpeg,.jpg,.png}
\title{Pré-rapport}
\author{Louisa~Bessad}%,~\IEEEmembership{Étudiante,~UPMC}
%\texttt{first.last@etu.upmc.fr}
\begin{document}
\maketitle
% Abstract
\begin{abstract}
Ce papier est un résumé conçernant le CGRA DART et les mécanismes de
détection/correction d'erreurs qu'il implémente. Nous verrons notamment
comment les concepteurs de DART ont pu améliorer de façon significative ses
performances grâce à un nouvelle technique de détection de type
\emph{on-line error-detecting} appelée \emph{residue code modulo 3}.
\end{abstract}
% Keywords
%\begin{IEEEkeywords}
%\end{IEEEkeywords}
% Content
\section{Introduction}

%% intro/objectif: virtualisation légère d'applications distribuées (tester des applications distribuées réelles: test regression et performance, légère car on veut tester des centaines d'instances)

%% Applications: stockage distribué (CEPH, TAHOE/LAFS) et RT event processing (Storm)

Dans le cadre de ce stage, nous allons nous intéresser aux applications
distribuées. Il s'agit d'applications dont une partie ou la totalité des
ressources n'est pas localisée sur la machine où l'application s'exécute, mais
sur plusieurs machines distinctes. Ces dernières communiquent entre elles via le
réseau pour s'échanger les données nécessaires à l'exécution de
l'application. Les applications distribuées ont de nombreux avantages: elles
permettent notamment d'augmenter la disponibilités des données en se les
échangeant, comme les applications Torrent (BitTorrent, $µ$Torrent...). Grâce au
projet BOINC\footnote{\url{https://boinc.berkeley.edu/}} par exemple, on peut
partager la puissance de calcul inutilisée de sa machine. Depuis une dizaine
d'années, la popularité de ces applications distribuées ne cesse de croître. On
peut notamment penser aux applications de stockage de données
LAFS\footnote{\url{https://tahoe-lafs.org/trac/tahoe-lafs}} et
CEPH\footnote{\url{http://ceph.com/}}. La première apporte un stockage robuste
qui préserve les données même si un site est physiquement détruit. La seconde
souhaite fournir performance, fiabilité et scalabilité.  Elles deviennent de
plus en plus complexes avec des contraintes et des exigences de plus en plus
fortes, en particulier au niveau des performances et de l'hétérogénéité des
plateformes et des ressources utilisées. Il est donc de plus en plus
difficile de créer de telles applications (absence d'horloge et mémoire
centrale, deadlock, race, famine) mais aussi de les tester.  En effet, malgré
l'évolution des applications distribuées, les protocoles d'évaluation de leurs
performances n'ont que peu évolués.
\newline

Actuellement, il existe trois façons principales de tester le comportement
d'applications distribuées \citep{gustedt2009experimental}: l'exécution sur
plate-forme réelle, la simulation et l'émulation.

La première solution consiste à exécuter réellement l'application sur un parc de
machines et d'étudier son comportement en conditions réelles. Cela permet de la
tester sur un grand nombre d'environnements. L'outil créé et développé en partie
en France pour cela est \textbf{Grid'5000}\footnote{Infrastructure de 8000 c\oe
  urs répartis dans la France entière crée en
  2005. \\ \url{https://www.grid5000.fr/mediawiki/index.php/Grid5000:Home}}\citep{GRID5000},
un autre outil développé à l'échelle mondiale est
\textbf{PlanetLab} \footnote{Crée en 2002, cette infrastructure de test compte
  aujourd'hui 1340 c\oe urs. \\ \url{http://www.planet-lab.org}}. Néanmoins,
pour mettre en \oe uvre ces solutions complexes, il faut disposer des
infrastructures nécessaires pour effectuer les tests. Il faut également écrire
une application complète capable de gérer toutes ces ressources
disponibles. Enfin, du fait du partage des différentes plateformes entre
plusieurs utilisateurs, les expériences sont difficilement reproductibles.

La seconde solution consiste à faire de la simulation: on modélise ce que l'on
souhaite étudier (application et/ou environnement) via un programme appelé
simulateur. Dans ce cas, pour pouvoir tester des applications distribuées sur un
simulateur, on doit d'abord abstraire l'application ainsi que l'environnement
d'exécution. Pour cela, on identifie les propriétés de l'application et de son
environnement puis on les transforme à l'aide de modèles mathématiques. Ainsi,
on va exécuter dans le simulateur le modèle de l'application et non
l'application réelle, dans un environnement également modélisé. Cette solution
est donc facilement reproductible, plus simple à mette en \oe uvre, et permet de
prédire l'évolution du système étudié grâce à l'utilisation de modèles
mathématiques. De nos jours, les simulateurs tel que
\textbf{SimGrid}\citep{CASANOVA:SimGrid, MARTIN:SimGrid} peuvent simuler des
applications distribuées mettant à contribution des milliers de
n\oe uds. Néanmoins, avec la simulation on ne peut valider qu'un modèle et non
l'application elle-même.

La troisième solution consiste à faire de l'émulation: on exécute réellement
l'application mais dans un environnement virtualisé grâce à un logiciel,
l'émulateur. Ce dernier joue le rôle d'intercepteur pour virtualiser
l'environnement d'exécution.
%On fera ainsi croire à l'application qu'elle s'exécute sur une machine autre que l'hôte.
Cette solution représente un intermédiaire entre la simulation et l'exécution
sur plateforme réelle visant à résoudre les limitations de ces deux
solutions. En effet, les actions de l'application sont réellement exécutées sur
la machine hôte (la machine réelle sur laquelle s'exécute l'émulation) et grâce
à la virtualisation, l'application pense être dans un environnement différent de
la machine réelle. De plus, cela évite d'avoir deux versions de l'application en
terme de code: une pour la simulation et une pour la production. L'émulation
peut-être faite \textit{off-line} (on sauvegarde les actions de l'application
sur disque et on les rejoue plus tard dans un simulateur) ou \textit{on-line}
(on bloque l'application le temps que le temps de réponse de la plateforme
virtualisée soit calculé).

%% Il existe deux types d'émulation pour les applications distribuées; la
%% virtualisation standard et la légère. On parle de virtualisation légère quand on
%% souhaite tester des applications sur une centaine d'instances.

Au cours de ce stage, nous allons nous intéresser plus particulièrement à l'exécution d'applications distribuées arbitraires au dessus de la plateforme de simulation SimGrid. Pour cela, nous allons présenter en section \ref{section:emulation} les méthodes utilisées
pour faire de la virtualisation %% légère: limitation et interception
. Puis en
section \ref{section:sota} nous présenterons les projets permettant de faire de
l'émulation pour tester des applications dans un environnement
distribué. Ensuite, nous expliquerons en section \ref{section:simterpose},
pourquoi dans le cadre du projet Simterpose c'est la virtualisation légère par
interception qui a été choisie et comment elle fonctionne. Pour finir, nous
concluerons en section \ref{section:ccl}.

\section{Pourquoi choisir l'émulation simulée}

Il existe trois façons de tester des applications distribuées. La première consiste à exécuter réellement l'application sur un parc de machines et d'étudier le comportement de l'application en temps-réel, ce que fait actuellement \textbf{Grid'5000} {\color{red}mettre citation}. Néanmoins pour mettre en \oe uvre cette solution complexe il faut disposer des architectures nécessaires pour effectuer les tests. De part le partage des différentes plateformes entre divers utilisateurs les expériences ne sont pas forcément reproductibles. Une deuxième solution consiste à simuler l'exécution des applications à l'aide d'un simulateur tel que \textbf{SIMGRID} {\color{red}mettre citation}. On doit alors modéliser l'application ainsi que l'environnements d'exécution grâce à des modèles mathématiques. Le problème avec la simulation est que l'on exécute pas vraiment l'application, on ne peut alors valider qu'un modèle et pas l'application puisqu'on réécrit l'application selon le modèle. Un des buts du projet étant de tester les applicationis sans avoir accès à leur code source, on ne peut donc pas choisir cette solution ou pas toute seule en tout cas. La troisième solution consiste à faire de l'émulation, c'est-à-dire que nous allons exécuter réellement l'applications mais dans en environnement virtuels. Ainsi nous aurons un simulateurs qui gérera l'environnement, l'application qui s'exécutera réellement sur la machine hôte et une API qui fera le lien entre l'application qui s'exécute et l'environnement simulé. On fera ainsi croire à l'application qu'elle s'exécute sur une machine autre que l'hôte. Il existe deux façons de faire de l'émulation: la dégradation et l'interception. Dans la première on rajoute la couche d'émulation au-dessus de la plateforme réelle (comme un hyperviseur pour une VM). Mais cela nous empêche d'émuler des machines plus puissante que l'hôte car le délai de réponse géré par l'émulateur ne peut-être inférieur à celui de l'hôte sinon l'hôte n'a pas le temps de faire les calculs nécessaires à l'application. Cette solution choisie par \textbf{Distem}{\color{red}mettre citation} est donc limitée à la capacité des plateformes à notre disposition. Dans le cas de l'interception pour faire croire à l'application qu'elle s'exécute sur une machine autre que l'hôte on va attraper toutes ses communications via une API du simulateur qui ensuite les transmetra au simulateur. Ce dernier s'occupera de calculer le temps de réponses de ces commumnications en se basant sur les performances de la machine qu'on est en train de simuler. Les calculs seront effectués sur la machine hôte mais le temps de réponse à l'application sera géré par le simulateur en fonction de l'architecture que l'on simule en utilisant le temps d'exécution de la machine hôte injectés dans le simulateur et en comparant les performances des deux architectures. Ainsi le temps de l'application sera celui du simulateur et non le temps réel. Cette solution est implémentée dans \textbf{Simterpose}{\color{red}mettre citation}. Dans notre cas d'émulation simulation, nous allons utiliser SIMGRID comme simulateur et Simterpose comme API de ce simulateur. Nous aurons donc Simterpose qui permettra d'utiliser SIMGRID avec des applications réelles en leur faisant croire qu'elles s'exécutent sur des machines distribuées. Maintenant que nous savons comment nous allons tester les applications distribuées nous allons voir comment fonctionne notre ``émulateur'' Simterpose.

\section{Comment fonctionne Simterpose}
\subsection{Interception des actions}
Simterpose est l'API qui permet d'intercepter les communications de
l'application avec la machine sur laquelle elle s'exécute. Sans cela
l'application se rendrait compte que l'environnement réel ne correspond pas à
celui dans lequel elle pense être. Une applications distribuées peut vouloir
communiquer avec l'hôte soit pour effectuer des calculs (SEB), soit pour
communiquer avec d'autres applications sur le réseau. Quand Simterpose
intercepte une communication venant de l'application, il modifie les
caractéristiques de cette dernière pour qu'elle puisse s'exécuter sur la machine
hôte. Quand la machine hôte renvoie une réponse à l'application, Simterpose
l'intercepte également et la modifie pour que l'application ne voit pas le
changement d'architecture. En même temps il envoie au simulateur des données
concernant le temps d'exécution de l'action sur la machine hôte pour calculer le
temps sur la machine simulé puis envoie ce temps à l'application en plus du
résultat pour mettre à jour son horloge. Ainsi les calculs sont réellement
exécuter sur la machine et le temps de réponse fournir par le simulateur qui va
influencer l'horloge de l'application permet d'imiter un environnement
distribué. 

{\color{red} MEttre un schéma de action interceptes, test modifie, renvoie,
  attrape reponse, simulateur, retour application}

Les délais calculés par le simulateur sont soit des temps de calculs soit des
temps de connextion. Les actions éxécutées par l'application peuvent-être soit
de simples calculs soit des requêtes de connexion ou communication réseau.

Pour intercepter ces actions, il faut d'abord choisir ce qu'on intercepte
exactement et avec quel outil. En effet une application peut communiquer avec
une machine via différentes abstractions.

{\color{red} schéma marion???} liste... tout seul tout seul puis le lien entre
les deux

Nous allons intercepter les actions que sont les AS fait par l'applciation, ainsi
nous sommes sûr d'intercepter tous les types de communications que
l'applicatione est susceptible d'initier avec le noyau. Les appels systèmes sont constitués de deux parties; la première, l'entrée, initialise l'appel via les registres de l'application qui contiennent les arguments de l'appel et donne la main au noyau. La seconde, la sortie, inscrit le reotur de l'appel système dans le registre de retour de l'application, les registres d'arguments contiennent toujours les valeurs reçues à l'entrée, et rend la main à l'application. Nous devons donc intercepter les deux parties de l'appel système pour maintenir notre environnement simulé et donc stopper l'application à chaque fois pour récupérer ou modifier les informations nécéssaires avant de lui rendre la main pour entrer ou sortir de l'appel système. Il existe de nombreux outils
en matières d'interception d'AS. On trouve notamment {\color{green} a remplir
  dire pourquoi abandonné}. Pour intercepter les éventuels AS d'une applciation
nous allons donc utiliser \textbf{ptrace}{\color{red}citation}. Cet outil qui
est lui-même un AS va permettre à un processus d'en contrôler un autre. Ainsi un
processus va dire qu'il souhaite être contrôle ``espionnée'' et un autre
processus le contrôlera ``processus espion''. Ce dernier va spécifier sur quels
évènements du processus ``espionné'' il veut être notifié, dans notre cas ce
sera les AS que l'on considérera comme point d'arrêt. \textit{Cet AS permet
  L'avantage de l'AS ptrace c'est qu'il permet d'écrire et de lire directement
  dans la mémoire des processus.} Ainsi quand l'application voudra faire un AS
quelconque elle sera bloquée et l'AS ptrace sera lancé et notifiera le
``processus espion'', ce dernier fera les modifications nécessaires dans les
registres de l'applications pour conserver la virtualisation de l'environnement
puis il rendra la main à l'application pour que l'AS ait lieu. Au retour de l'AS
le ``processus espionné'' sera de noveau stopper et un ptrace sera envoye à
l''espion'' qui remodifiera les informations nécessaires et rendra la main à
l'application qui sortira de son AS avec un résultat exécuté sur la machine hôte
et un temps d'exécution et une horloge fournie par le simulateur. Néanmoins il a
été montré dans un précédent stage que l'AS ptrace est inefficace voir inutile
en ce qui concerne tous les AS temporel que l'application voudrait
faire. \textit{Les AS ``time'', ``clock\_gettime'', ``gettimeofday'' avec
  \textbf{ptrace} ne sont pas possibles, d'où l'alliance avec LD\_PRELOAD} Par
exemple lors d'un gettimeofday l'AS n'est pas lancé on répond directement au
niveau de la bibliothèque ainsi on n'arrive même pas au niveau de l'AS, donc
ptrace ne fait rien.

On pourrait alors se dire qu'une bonne solution serait d'intercepter les actions
de l'application au niveau des bibliothèques pour cela il existe l'éditeur de
lien dynamique \textbf{LD\_PRELOAD} qui permet de précharger des bibliothèqeus{\color{red} rajouter des trucs sur
  VDSO}. Ainsi on va créer notre propre bibliothèque de fonction pour chaque
fonction que l'application serait susceptible d'utiliser. Ainsi dans notre
nouvelle fonction on ajoutera toutes les modifications nécessaires pour
maintenir notre environnement simulé puis on fera appel à la vraie fonction
celle de base et on préchargera cette bibliothèque avant les autres grâce à
LD\_PRELOAD. Ainsi nos fonctions passeront avant les fonctions de base de
l'OS. Le problème de cette solution est que si l'application fait un appel à une
fonction que l'on n'a pas réécrit rien ne l'empchera de faire l'AS directement
sans avoir modifier ses paramètres, de fait cette solution n'est pas
complète. 

Ainsi on peut voir que LD\_PRELOAD résout les problèmes de ptrace et
inversement, une solution choisie lors d'un précédent stage est donc d'allier
les deux. On surchargera les fonctions temporelles dans notre bibliothèque préchargée avec LD\_PRELOAD pour pallier les lacunes temporelles de ptrace à ce sujet. Et pour toutes les autres fonctions ptrace s'en occupera ainsi on est certain de n'oublier aucune fonction. Maintenant que nous savons ce que nous devons intercepter et comment
l'intercepter nous allons voir ce que nous devons modifier pour pouvoir
maintenit notre émulation simulation.

\textit{Pour faire face à ce problème il a été choisi d'utiliser l'éditeur de
  lien dynamique \textbf{LD\_PRELOAD}, ce dernier intercepte les appels de
  l'application au niveau des bibliothèques. Pour cela on va créer une
  bibliothèque.}

\subsection{Modification à effectuer}
Lorsque ptrace est appelé en entrée ou sortie d'AS, les modifications à apporter ne sont pas forcément les mêmes selon qu'il s'agit d'une action nécessitant l'utilisation du réseau ou non. Dans le cas d'un simple calcul ce qu'il faut maintenir pour l'application c'est une vision du temps correspondant à celle qui s'écoulerait si elle était vraiment sur la machine simule. Ainsi en entrée d'appel système on n'a pas besoin de modifier quoique ce soit, par contre au retour il faut modifier le temps d'exécution du calcul en le remplaçant par celui calculé par le simulateur. Dans le cas d'une communication réseau il y a beaucoup plus de travail. En effet l'application a une vision d'un réseau qui ne correspond pas au réseau local ainsi quand elle veut faire une communication réseau la machine qu'elle cherche à joindre via une adresse et des ports possède en réalité une autre adresse et d'autres ports en local. Il faut donc en entrée de l'appel système remplacé l'adresse ``virtuelle'' par l'adresse réelle sur le réseau local pour que l'appel système puisse se faire vers sur la bonne destination. Au retour de l'appel système il faut remodifier les paramètres en remettant la fausse adresse virutlle à la place de la réelle pour que l'application pense toujours être dans son environnement simulé. Pour gérer ces modifications deux solutions ont été proposées lors d'un précédent stage: la  ``médiation par traduction d'adresse'' et la ``full mediation''.
\paragraph{Traduction d'adresse} Avec ce type de médiation on considère que le noyau gère des communications ainsi en entrée et sortie d'appel système Simterpose va juste modifier les adresses et ports de connexion ou de communication via un tableau de correspondance qu'il gère. Pour lire et modifier les paramètres de l'appel système on peut soit modifier les registres du processus avant l'entrée et la sortie de l'appel via ptrace {\color{red}ou on peut directement écrire dans le /proc/pid/mem}
\paragraph{Full mediation} Dans ce cas le noyau ne va plus gérer des communications car nous allons empêcher l'application d'établir des connexions avec une autre application via des sockets. Quand l'application voudra faire un appel système de type communication vers une autre applciation, le processus espion de Simterpose qui sera notifié via ptrace neutralisera l'appel système. Ensuite ce processus en utilisant ptrace récupérera en lisant dans la mémoire du processus espionné les ifnormations à envoyer ou récupérer et ira directement lire ou écrire ces informations dans la mémoire du destinataire. Ainsi on n'a pas besoin de gérer de tableau de correspondance d'adresse et de ports et les applications peuvent conserver les adresses simulées qu'elles considèrent comme réelles.
Même si la ``full mediation'' permet d'éviter les communications réseaux et de conserver des tables de correspondances, dans le cas d'applications qui communiquent énormément elle s'avère moins efficace. En effet les coût d'appels mémoires sont plus élevés que ceux des communications réseaux.

% Bib
\begin{thebibliography}{1}
\end{thebibliography}
\end{document}
