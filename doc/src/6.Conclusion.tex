\section{Conclusion}
Dans ce rapport, nous avons présenté notre objectif ainsi qu'un état de l'art
des différentes approches et outils existants. Pour pouvoir exécuter n'importe
quel type d'applications distribuées dans des conditions environnementales qui
permettront de les étudier, la meilleure solution semble être la virtualisation
par interception. Les projets existants ne permettant pas de résoudre les quatre
problèmes engendrés par cette virtualisation (gestion du temps, des threads, des
communications réseaux et le DNS), un nouveau projet a été lancé.

L'émulateur Simterpose dévelopé au LORIA permet d'exécuter et de tester des
applications distribuées réelles sans disposer de leur code sources dans un
environnement distribué virtuel. Il se base sur la plateforme de simulation
Simgrid pour mettre en place l'environnement d'exécution dans lequel
l'application pensera s'exécuter. Pour maintenir la virtualisation, les actions
des applications sont interceptées et modifiées pour ensuite être exécutées. On utilise SimGid pour calculer la réponse de l'environnement virtuel aux
différentes actions. La solution proposée intercepte les actions à deux niveaux
différents, appels systèmes et des bibliothèques, afin de ne pas en oublier. Simterpose permet également d'injecter diverses fautes dans la
simulation pour avoir une virtualisation plus réaliste.

Néanmoins, il reste à ajouter certaines fonctionalités (DNS, fonctions de temps)
afin de pouvoir vérifier qu'il est possible de mettre en place une
virtualisation par inteception qui apporte une réponse à notre
problématique. Nous projettons par la suite d'évaluer les performances de notre
projet, afin de le modifier pour le rendre plus efficace et de maximiser la
taille des expériences.

