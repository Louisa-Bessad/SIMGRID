\subsection{Expériences}
Dans la section \ref{section:evaluation}, nous avons présenté les différents tests que nous avons réalisés. Pour nos expériences, nous avons utilisé des tailles de message fixe. Il pourrait être intéressant de faire des tests en tirant aléatoirement la taille des messages entre X octets Y Méga-octets. De plus, pour terminer ce stage, nous souhaitions faire exécuter à Simterpose une application de torrent, telle que BitTorrent, mais nous avons manqué de temps et cela reste à faire.

Lors de la rédaction de ce rapport, nous avons également réfléchi à de nouvelles expériences. Le nombre de processus en cours d'exécution influant forcément sur les performances, on pourrait exécuter plusieurs processus ``espions'' en parallèle afin de voir comment évoluent les performances de Simterpose. Ainsi, nous pourrions savoir  combien de processus ``espions'' peuvent être lancés simultanément et donc évaluer la scalabilité de notre émulateur. Dans ce cas, il serait intéressant de mesurer l'impact d'une telle exécution sur la mémoire en plus du temps d'exécution. Cela nous permettrait également d'évaluer encore plus finement les deux types de médiations implémentées. Par exemple, on pourrait voir si l'avantage de l'\textit{address translation} est accru ou décru quand des centaines de processus l'utilisent simultanément.

De plus, une fois que nous aurons un émulateur complet, nous vérifierons qu'il est toujours possible de répondre à notre problématique de virtualisation légère. Puis, nous pourrons essayer d'améliorer son efficacité en fonctions des choix d'implémentations définitifs qui auront été faits.


