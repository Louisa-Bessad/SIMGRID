\section{Simterpose: la médiation}
\label{section:simterpose}

Dans le cadre du projet Simterpose de virtualisation légère et de test
d'applications distribuées, c'est l'émulation par interception qui a été
choisie. En effet, le but final étant de pouvoir évaluer n'importe quelle
application distribuée sur n'importe quel type d'architecture, on peut se
retrouver à devoir émuler des machines plus puissantes que l'hôte, ce que
l'émulation par dégradation ne permet pas. Pour cela, on va utiliser SIMGRID
comme plateforme de simulation et Simterpose comme émulateur Dans cette section
nous allons voir pourquoi ce choix ainsi que le fonctionnement interne de cet
émulateur, notamment les outils préséntés en section \ref{section:emulation}
qu'il utilise.

