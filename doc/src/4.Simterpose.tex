
\section{Simterpose: Symbôles sur lesquels faire de la médiation}
\label{section:simterpose}
Dans le cadre du projet Simterpose de virtualisation légère et de test
d'applications distribuées, c'est l'émulation par interception qui a
été choisi. En effet le but final étant de pouvoir évaluer n'importe
quelle application distribuée sur n'importe quel type d'architecture,
on pourrait se retrouver à devoir émuler des machines plus puissantes
que l'hôte, ce que l'émulation par dégradation ne permet
pas. {\color{red}Pour faire cette émulation on va utiliser SIMGRID
  comme simulateur et Simterpose comme API pour ce simulateur.Il
  jouera donc le rôle d'émulateur en nous permettant d'utiliser
  SIMGRID avec des applications réelles tout en leur faisant croire
  qu'elles s'exécutent sur des machines distinctes. Simterpose étant
  l'API qui va nous permettre d'intercepter les communications de
  l'application avec la machine sur laquelle elle s'exécute et de
  faire de la médiation nous allons étudier son fonctionnement.}

\subsection{Organisation générale}
%schéma tableau

\subsection{Les communications réseaux}
 %-> syscall -> ptrace (full mediation, address translation)

Lorsque ptrace est appelé en entrée ou sortie d'appel système, les
modifications à apporter ne sont pas forcément les mêmes selon qu'il
s'agit d'une action nécessitant l'utilisation du réseau ou non. Dans
le cas d'un simple calcul ce qu'il faut maintenir pour l'application,
c'est une vision du temps correspondant à celle qui s'écoulerait si
elle était vraiment sur la machine simulée. Ainsi en entrée d'appel
système on n'a pas besoin de modifier quoique ce soit, par contre au
retour il faut modifier le temps d'exécution du calcul en le
remplaçant par celui calculé par le simulateur.

Dans le cas d'une communication réseau il faut gérer la transition
entre réseau local et réseau simulé. En effet l'application possède
une adresse IP et des numéros de ports ``virtuels'' qui ne
correspondent pas forcément à ceux attribués dans le réseau local. De
plus on ne peut pas se baser uniquement sur les numéros de
\textit{file descriptor} associé à une socket pour identifier deux
entités qui communiquent entre elles.En effet ces \textit{file
  descriptor} sont uniques pour chaque socket d'un processus, mais
plusieurs processus peuvent avoir un même numéro de \textit{file
  descriptor} pour des sockets de communicatiosn différentes puisque
chacune à son propre espace mémoire. Pour pallier à ce probème on va
utiliser en plus du numéro de socket, les adresses IP et les ports
locaux et distants des deux entités qui souhaitent communiquer comme
moyen d'identification. Pour gérer toutes ces modifications deux
solutions ont été proposées lors d'un précédent stage
\cite{interception:GUILLAUME:interception_syscall}: la ``médiation par
traduction d'adresse'' et la ``full mediation''.

{\color{red}schéma}
\paragraph{Traduction d'adresse}
 Avec ce type de médiation on considère que le noyau gère des
 communications. Ainsi en entrée et sortie d'appel système Simterpose va juste
 s'occuper de la transition entre réseau ``virtuel'' et réseau local. Pour cela
 Simterpose gère un tableau de correspondance, dans lequel pour chaque
 application on a un couple (adresse IP et des ports ``virtuels'', adresse IP et
 ports ``réel'' sur le réseau).  De fait en entrée de l'appel système,
 Simterpose devra remplacer l'adresse et les ports ``virtuels'' de l'application
 par l'adresse et les ports réels sur le réseau local, ainsi l'appel système se
 fera avec une source qui existe réellement sur le réseau. Au retour de l'appel
 système il faudra remodifier les paramètres en remettant l'adresse et les ports
 ``virtuels'' pour que l'application pense toujours être dans son environnement
 simulé.  La limite de cette approche est lié au nombre de port disponibles sur
 l'hôte. 

\paragraph{Full mediation} 
Dans ce cas le noyau ne va plus gérer des communications car nous allons
empêcher l'application d'établir des connexions avec une autre application via
des sockets. Il n'y aura ni socket ni communication. Quand l'application voudra
faire un appel système de type communication vers une autre applciation, le
processus espion de Simterpose qui sera notifié via ptrace neutralisera l'appel
système. Ensuite ce processus en utilisant ptrace récupérera en lisant dans la
mémoire du processus espionné les ifnormations à envoyer ou récupérer et ira
directement lire ou écrire ces informations dans la mémoire du
destinataire. Ainsi on n'a pas besoin de gérer de tableau de correspondance
d'adresse et de ports et les applications peuvent conserver les adresses
simulées qu'elles considèrent comme réelles.  Même si la ``full mediation''
permet d'éviter les communications réseaux et de conserver des tables de
correspondances, dans le cas d'applications qui communiquent énormément et
utilisent de grosses données elle s'avère moins efficace. En effet les appels à
mémoires sont bien plus couteux que les communications réseaux.

\subsection{Les thread}
 %% syscall clone + libcalls

\subsection{Le temps}
 %% -> syscall (- system wide), VDSO-linker (cross process ou VDSO)

\subsection{DNS}
%% libcalls (ne rien rater), config fake (system wide), intercept 53 ( plus dur que nécessaire, port dns autre ou pas)
