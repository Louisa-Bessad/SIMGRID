\subsubsection{Action sur le fichier source}
\label{section:source}
%% reimplem SMPI (trop spé) ,source to source/ pass LLVM( gcc+libc=consanguin) 
%% , Coccinelle
Le premier niveau auquel on peut se placer pour intercepter les actions est le fichier source de l'application. On pourrait avant de compiler le code réécrire les parties qui nous intéressent.

Un premier outil pour cela est le programme Coccinelle \citep{cocci}. Il permet de trouver et transormer automatiquement des parties spécifiques d'un code source C. Pour cela, Coccinelle fournit le langage SmPL\footnote{Semantic Patch Language} permettant d'écrire les patchs sur lesquels il va se baser pour transformer le code. Un patch contient une suite de règles, chacune transforme le source en ajoutant ou supprimant du code. Lors de son exécution, Coccinelle scanne le code et cherche les lignes qui satisfont les conditions des règles spécifiées dans le patch et applique les transformations correspondantes. Dans notre cas, il s'agirait de toutes les actions de communications directes ou indirectes avec le noyau susceptibles de mettre à jour l'environnement virtuel. Néanmoins, il ne faut pas oublier de définir une règle pour chacune de ces actions sinon l'interception sera contournée. De plus, il faut pouvoir accéder au fichier source pour le modifier, or cela n'est pas toujours possible.

Une seconde solution beaucoup plus spécifique est de réimplémenter totalement le programme SMPI\footnote{Simluation d'appplications MPI} \citep{SMPI, clauss2011single} qui permet de simuler des applications MPI. Si dans son implémentation on modifie la façon de gérer les communications on pourrait mettre en place et maintenir notre environnement virtuel. Pour cela, il devra modifier chaque communication reçue qui pourrait mettre en péril le maintien de notre virtualisation.

