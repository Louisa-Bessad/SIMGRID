\documentclass[a4paper,12pt]{article}

\usepackage[T1]{fontenc}
\usepackage{xltxtra}
\usepackage[francais]{babel}
\usepackage{fancyhdr}

\usepackage{epsfig}
\usepackage{calc}
\usepackage{url}
\usepackage{boxedminipage}

\usepackage{titlesec}
%Pour enlever cette merde de chapter!
%\titleformat{\chapter}[hang]{\bf\huge}{\thechapter}{2pc}{}
\usepackage{graphicx}
\usepackage{color}
\usepackage{float}

\usepackage{tabularx}
\usepackage[hidelinks]{hyperref}
\usepackage[sort, authoryear]{natbib}
\usepackage{subcaption}
\bibpunct{[}{]}{,}{n}{,}{,}

% --------------------------------
%Pour compiler sur Ubuntu
%%\usepackage[latin1]{inputenc}   

% --------------------------------
%Pour compiler sur Manjaro        
\usepackage[utf8]{inputenc}
\usepackage{algorithmicx}
\usepackage{algpseudocode}

% --------------------------------
% Fonction of available fonts
%\usepackage{fontspec} 
%\setmainfont{Adobe Garamond Pro}       
%\setmainfont{EB Garamond}
% --------------------------------

%\usepackage[language=french]{csquotes}



%%%%%%%%%%%%%%%%%%%%%%%%%%%%%%%%%%%%%%%%%%%%%%%%%%%%%%%%%%%
%%%%%%%%%%%%%%%%%%%%%%%%%%%%%%%%%%%%%%%%%%%%%%%%%%%%%%%%%%%
%% Définitions à personnaliser 

%% Pour les noms, utilisez la première lettre du prénom suivi du 
%% nom de famille (première lettre majuscule, reste en minuscule).


%%%% Indiquer le nom de l'encadrant ci-dessous:

\def\nomEncad{Martin \textsc{Quinson}}

%% Si le projet est co-encadré indiquer les deux noms à la suite dans 
%% Le même champs


%%%% Indiquer les noms des étudiants participant ci-dessous:

\def\nomEtudA{Louisa \textsc{Bessad}}

%% Si le projet est encadré par moins de 4 étudiants laissez
%% les variables inutiles vides


%%%% Indiquer la référence (numero) et le nom du sujet ci-dessous:

\def\refProjet{Numéro Projet} 
\def\titreProjetCourt{Titre court}
\def\titreProjetLong{Titre long}

%% Le titre court ne doit pas faire plus d'une vingtaine de caractère
%% résumez le à quelques mots essenciels


%%%% Indiquer le type de document et sa version ci-dessous:

\def\typeDoc{Pré-rapport}
 
%% - Rapport intermédaire
%% - Rapport final

%\let\origsec\section
%\renewcommand{\section}[1]{\newpage\origsec{#1}}



%%%%%%%%%%%%%%%%%%%%%%%%%%%%%%%%%%%%%%%%%%%%%%%%%%%%%%%%%%%
%%%%%%%%%%%%%%%%%%%%%%%%%%%%%%%%%%%%%%%%%%%%%%%%%%%%%%%%%%%
%% Définitions à ne pas modifier
 
%%%% ||| Mise en page verticale ||| 
\setlength{\voffset}{-1in} % a4:reste 297mm pour les 5 suivants:
\setlength{\topmargin}{15mm}         % avant l'en-tête
\setlength{\headheight}{20mm}        % hauteur de l'en-tête 
\setlength{\headsep}{10mm}            % entre l'en-tête et le corps
\setlength{\textheight}{220mm}       % hauteur du corps
\setlength{\footskip}{12mm}          % pied de page par rapport au corps 
%\setlength{\footlength}{2em}

%%%%% --- Mise en page horizontale ---
\setlength{\hoffset}{-1in} % a4:reste 210mm 
\setlength{\oddsidemargin}{25mm}     % entre hoffset et le corps
\setlength{\evensidemargin}{25mm}    % entre hoffset et le corps
\setlength{\marginparwidth}{0mm}     % largeur de la marge
\setlength{\marginparsep}{0mm}       % séparateur corps marge
\setlength{\textwidth}{160mm}        % largeur du corps

%\usepackage{fullpage}
%\setlength{\headheight}{20mm}        % hauteur de l'en-tête 
%\setlength{\headsep}{10mm}            % entre l'en-tête et le corps
%\setlength{\textheight}{200mm}
%\setlength{\footskip}{0mm}          % pied de page par rapport au corps 

\def\annee{2015}



%%%%%%%%%%%%%%%%%%%%%%%%%%%%%%%%%%%%%%%%%%%%%%%%%%%%%%%%%%%
%% Début du document

\begin{document}

\selectlanguage{francais}



%%%%%%%%%%%%%%%%%%%%%%%%%%%%%%%%%%%%%%%%%%%%%%%%%%%%%%%%%%%
%% Définition des en-têtes et pied de pages
\pagestyle{fancyplain}
%\fancyhead{}
%\fancyfoot{}
%
\fancyhead[L]{\textsc{Université Pierre et Marie Curie}\\
          Master Informatique\\ Stage M2 \annee \\ \nomEtudA}
\fancyhead[C]{\textbf{Pré-rapport}}%\\\titreProjetCourt}}
\fancyhead[R]{\textsc{LORIA} \\ {\color{white} b} \\ {\color{white} b} \\ \nomEncad}

\fancyfoot[L]{\includegraphics[width=4cm]{Pictures/png/UPMC_sorbonne.png}}
\fancyfoot[C]{\textbf{\thepage/\pageref{fin}}}
\fancyfoot[R]{\includegraphics[width=4cm]{Pictures/loria_logo.jpg}}

%\lhead[\fancyplain{}{\texttt{Université Pierre et Marie Curie}\\
%          Master Informatique\\ UE \textbf{PSAR} fév. \annee \\ \nomEncad}]
%      {\fancyplain{}{\textsc{Université Pierre et Marie Curie}\\
%          Master Informatique\\ UE \textbf{PSAR} \annee \\ \nomEncad}}
%\chead[\fancyplain{}{\textbf{Projet \refProjet\\\titreProjetCourt}}]
%      {\fancyplain{}{\textbf{Projet \refProjet\\\titreProjetCourt}}}
%\rhead[\fancyplain{}{\nomEtudA\\\nomEtudB}]
%      {\fancyplain{}{\nomEtudA\\\nomEtudB}}
%\lfoot[\fancyplain{}{\epsfig{figure=UPMC_sorbonne.eps,width=3cm}}]
%      {\fancyplain{}{\epsfig{figure=UPMC_sorbonne.eps,width=3cm}}}
%\cfoot[\fancyplain{}{\textbf{\thepage/\pageref{fin}}}]
%      {\fancyplain{}{\textbf{\thepage/\pageref{fin}}}}
%\rfoot[\fancyplain{}{\typeDoc}]
%      {\fancyplain{}{\typeDoc}}


%%%%%%%%%%%%%%%%%%%%%%%%%%%%%%%%%%%%%%%%%%%%%%%%%%%%%%%%%%%

      \begin{center}
        \begin{boxedminipage}{12cm}{
            \begin{center}
              ~\\\LARGE\textbf{\titreProjetLong}\\
              ~\\\large Étudiante: \textbf{\nomEtudA,}\\
              ~\\\large Encadrant: \textbf{\nomEncad}\\
              ~
            \end{center}
            }
        \end{boxedminipage}
      \end{center}

%\vfill

\newpage

\tableofcontents
%\vfill
\newpage

%Abstract
\begin{abstract}
 TODO
  %% Dans le cadre de ce stage nous allons nous intéresser aux applications ditribuées à large échelle et comment on peut les tester et évaluer leurs performances via une combinaison d'émulation et de simulation en utilisant SIMGRID et Simterpose qui sont deux projets européens. SIMGRID a été lancé en 1999 pour étudier des algorithmes d'ordonnancement sur des plateformes hétérogènes dans un environnement distribué et faciliter leur programmation. Il fournit les outils de base nécessaire à la simulation de ce type d'applications. Simterpose s'insère dans le projet SIMGRID afin de pouvoir étudier des applications complètes et pas uniquement leur modèle que l'on fournit habituellement en paramètre au simulateur. Le but est de faire de l'émulation en utilisant un simulateur que sera SIMGRID. Puisque nous nous intéressons aux applications distribuées notre émulateur doit pouvoir\textit{i)} exécuter un grand nombre d'instances d'une même application sur un même système afin de pouvoir debugguer, \textit{ii)} évaluer des applications ayant de nombreuses condition d'exécution (simple n\oe ud, réseau complet), \textit{iii)} collecter les informations concernant l'application pendant qu'elle s'exécute.
\end{abstract}
\newpage

% Content
%% {\color{white} blabla} \vspace{7cm}


Je voudrais remercier mon tuteur Martin Quinson pour m'avoir permis d'effectuer ce stage, durant lequel j'ai beaucoup appris.

Je tiens également à remercier mon encadrant universitaire Sébastien Monnet pour ses précieux conseils.

%% \newpage
\section{Introduction}

%% intro/objectif: virtualisation légère d'applications distribuées (tester des applications distribuées réelles: test regression et performance, légère car on veut tester des centaines d'instances)

%% Applications: stockage distribué (CEPH, TAHOE/LAFS) et RT event processing (Storm)

Dans le cadre de ce stage, nous allons nous intéresser aux applications distribuées. \textit{ Autrement dit aux} applications dont une partie ou la totalité des ressources  n'est pas stockée sur la machine où l'application s'exécute, mais sur plusieurs machines distinctes. Ces dernières communiquent entre elles via le réseau pour s'échanger les données nécessaires à l'exécution de l'application. Les applications distribuées ont de nombreux avantages; elles permettent notamment d'augmenter la disponibilités des données en se les échangeant,\textit{ comme les applications Torrent (BitTorrent, $µ$Torrent...)}. Grâce au projet BOINC\footnote{\url{https://boinc.berkeley.edu/}} par exemple, on peut partager la puissance de calcul inutilisée de sa machine. Depuis une dizaine d'années la popularité de ces applications distribuées ne cesse de croître. Elles deviennent de plus en plus complexes avec des contraintes et des exigences de plus en plus fortes, en particulier au niveau des performances et de l'hétérogénéité des plate-formes et des ressources utilisées. Il devient donc de plus en plus difficiles de créer de telles applications mais aussi de les tester. En effet, malgré l'évolution des applications distribuées, les protocoles d'évaluation de leurs performances n'ont que peu évolués.

Actuellement, il existe trois façons de tester le comportement d'applications distribuées; l'exécution sur plate-forme réelle, la simulation et l'émulation. 

La première solution consiste à exécuter réellement l'application sur un parc de machines et d'étudier son comportement en temps-réel. Cela permet de la tester sur un grand nombre d'environnement. L'outil créé et développé en partie en France pour nous permettre de faire cela est \textbf{Grid'5000}\footnote{Infrastructure de 8000 c\oe urs répartis dans la France entière crée en 2005. \\ \url{https://www.grid5000.fr/mediawiki/index.php/Grid5000:Home}}\citet{GRID5000}, un autre outil développé à l'échelle mondiale est \textbf{PlanetLab} \footnote{Crée en 2002, cette infrastructure de test compte aujourd'hui 1340 noeuds. \\ \url{http://www.planet-lab.org}}. Néanmoins pour mettre en \oe uvre ces solutions complexes, il faut disposer des infrastructures nécessaires pour effectuer les tests. Il faut également écrire une application capable de gérer toutes ces ressources disponibles. De plus, du fait du partage des différentes plate-formes entre plusieurs utilisateurs, les expériences ne sont pas forcément reproductibles. 

{\color{red}\textit{La seconde solution consiste à faire de la simulation, c'est-à-dire à utiliser un programme appelé simulateur pour nous permettre de simuler ce que l'on souhaite étudier.}} Dans notre cas, pour pouvoir tester des applications distribuées sur un simulateur, on doit d'abord représenter de façon théorique l'application ainsi que l'environnement d'exécution. Pour cela, on identifie les propriétés de l'application et de son environnement puis on les transforme à l'aide de modèles mathématiques. Ainsi, on va exécuter dans le simulateur le modèle de l'application dans un environnement également modélisé et non l'application réelle. Cette solution est donc facilement reproductible, simple à mettre en \oe uvre, \textit{ quand on sait} modéliser l'application, et permet de prédire l'évolution du système étudié grâce à l'utilisation de modèles mathématiques. De nos jours, les simulateurs, tel que  \textbf{SIMGRID}\citet{SIMULATIONCASANOVA, SIMULATIONMARTIN}, peuvent simuler des applications distribuées mettant à contribution des milliers de noeuds. Néanmoins, avec la simulation on ne peut valider qu'un modèle et pas l'application elle même puisqu'on exécute seulement un modèle. 

La troisième solution consiste à faire de l'émulation, cela signifie que nous allons exécuter réellement l'application mais dans un environnement virtualisé grâce à un logiciel, l'émulateur. Ce dernier joue le rôle d'intercepteur et utilise un simulateur pour virtualiser l'environnement d'exécution.
%On fera ainsi croire à l'application qu'elle s'exécute sur une machine autre que l'hôte.
Cette solution représente un intermédiaire entre la simulation et l'exécution sur plate-forme réelle visant à résoudre les limitations de ces deux solutions. En effet, les actions de l'application sont réellement exécutées sur la machine hôte, autrement dit la machine réelle sur laquelle s'exécute l'émulation. Mais on fait croire à l'application grâce au simulateur qu'elle se trouve dans un environnement différent de la machine \textit{réelle}. De plus, cela évite d'avoir deux versions de l'application en terme de code: une pour la simulation et une pour la production. Dans notre cas l'émulation peut-être faite \textit{off-line}; on sauvegarde les actions de l'application sur disque et on les rejoue plus tard dans le simulateur ou \textit{on-line}; \textit{ on bloque l'application le temps que les actions soient reportées dans le simulateur pour qu'il calcule le temps de réponse de la plate-forme simulée}.

Dans le cadre du projet Simterpose c'est l'émulation qui a été choisie pour tester des applications distribuées. En effet la simulation n'était pas une bonne solution puisque nous voulons valider les applications et non leur modèle. En ce qui concerne l'exécution sur plate-forme réelle, il y avait trop de contraintes matérielles à satisfaire.

{\color{red}TODO: Transition}
Il existe deux types d'émulation pour les applications distribuées; la virtualisation standard et la ``légère''. On parle de virtualisation ``légère'' quand on souhaite tester des applications sur une centaine d'instances. Dans ce rapport nous allons présenter en section \ref{section:emulation} les méthodes utilisées pour faire de la virtualisation légère: limitation et interception. Puis en section \ref{section:sota} nous verrons les projets qui existent aujourd'hui pour ce type de virtualisation. Pour finir en section \ref{section:simterpose} nous expliquerons pourquoi dans le cadre du projet Simterpose c'est la virtualisation légère par interception qui a été choisie et comment elle fonctionne.

\newpage
\section{Méthodes possibles pour la virtualisation légère}
\subsection{Emulation par limitation / Virtualisationi standard}
\begin{itemize}
\item principe: limiter l'accès aux ressources par exemple (cgroup, netstat, cpuburner), temps d'un SEB (bench avec netlink, limiter (cap))
\item avantage plus simple
\item désavantages: host>>target, modèle à vérifier, contrôle expérimental fin
\end{itemize}

\subsection{Emulation par interception}
 principe: interception des actions et médiation (pas juste intereption et rejeu). Intercepter des symbôles pour en changer l'effet

\subsubsection{Action sur le fichier source}
reimplem SMPI ( trop spé) ,source to source/ pass LLVM( gcc+libc=consanguin) 

\subsubsection{Action sur le binaire}
Valgrind (perf pourrie)

\subsubsection{Médiation directe des appels de fonctions}
pourquoi: pthread, temps

\paragraph{linker: LD\_PRELOAD}
pas suid
%% On pourrait alors penser qu'une bonne solution serait d'intercepter les actions
%% de l'application au niveau des bibliothèques.{\color{red} rajouter des trucs sur
%%   VDSO}. Pour cela il existe la variable d'environnement \textbf{LD\_PRELOAD}
%% qui contient la liste des bibliothèques à précharger et qui est utilisée par le
%% noyau lors du premier lancement d'un programme. En effet par défaut Linux
%% effectue une édition de lien dynamiqe, l'édition de lien statique n'étant
%% choisie qu'en l'absence de bibliothèques partagées définissant les fonctions
%% utilisées par l'application. On va donc créer notre propre bibliothèque de
%% fonctions surchargeant chaque fonction susceptible d'être utilisée par
%% l'application. Une fonction surchargée contiendra alors toutes les modifications
%% nécessaires pour maintenir notre environnement simulé suivi de l'appel à la
%% fonction initiale puisqu'on souhaite juste intercepter l'appel et pas
%% l'empêcher. On préchargera cette bibliothèque avant les autres en la plaçant
%% dans la variable LD\_PRELOAD, ainsi nos fonctions passeront avant les fonctions
%% des bibliothèques usuelles.

%% Néanmoins si l'application fait un appel système directement sans passer par la
%% couche \textit{Bibliothèques} Fig.~\ref{AS_Communication} notre mécanisme
%% d'interception est countournée. En effet on ne peut surcharger que des fonctions
%% avec cette solution, pas des appels sytèmes. De même si on oublie de réécrire
%% une fonction d'une des bibliothèques utilisée par l'application. Cette solution
%% n'est donc pas suffisante pour le modèle d'interception que nous souhaitons
%% avoir.

%% Cependant on peut voir que LD\_PRELOAD résout les problèmes de ptrace concernant
%% les fonctions de temps, et inversement puisque ptrace permet d'intercepter les
%% appels systèmes que le modèle d'interception avec LD\_PRELOAD ne permet pas de
%% gérer. Une solution choisie lors d'un précédent stage est donc d'allier les
%% deux. On surchargera les fonctions temporelles dans notre bibliothèque
%% préchargée avec LD\_PRELOAD pour pallier les lacunes temporelles de ptrace. Et
%% pour toutes les autres fonctions ptrace s'en occupera, ainsi on est certain de
%% n'oublier aucune fonction. Maintenant que nous savons ce que nous devons
%% intercepter et comment l'intercepter nous allons voir ce que nous devons
%% modifier pour pouvoir maintenit notre émulation simulation.

%% \textit{Pour faire face à ce problème il a été choisi d'utiliser l'éditeur de
%%   lien dynamique \textbf{LD\_PRELOAD}, ce dernier intercepte les appels de
%%   l'application au niveau des bibliothèques. Pour cela on va créer une
%%   bibliothèque.}

\paragraph{linker got injection}
plus dur que nécessaire

\subsubsection{Médiation des Appels Système}
pourquoi: read/write, comm reseau

\paragraph{ptrace}
%% Nous allons intercepter les actions que sont les appels systèmes faits par
%% l'applciation, ainsi nous sommes sûrs d'intercepter tous les types de
%% communications que l'application est susceptible d'initier avec le noyau. Les
%% appels systèmes sont constitués de deux parties; la première, l'entrée,
%% initialise l'appel via les registres de l'application qui contiennent les
%% arguments de l'appel puis donne la main au noyau. La seconde, la sortie, inscrit
%% le retour de l'appel système dans le registre de retour de l'application, les
%% registres d'arguments contennant toujours les valeurs reçues à l'entrée de
%% l'appel système, et rend la main à l'application. Nous devons donc intercepter
%% les deux parties de l'appel système pour maintenir notre environnement simulé et
%% donc stopper l'application à chaque fois pour récupérer ou modifier les
%% informations nécéssaires avant de lui rendre la main pour entrer ou sortir de
%% l'appel système.

%% Pour faire cela il existe de nombreux outils de différents types. Il y a l'API
%% ptrace, qui est lui même un appel système et qui permet de tracer tous les
%% événement désirés d'un processus contrôlé. Néanmoins ce dernier fait de nombreux
%% changements de contexte pour intercepter des événements. De plus il gère mal les
%% processus quand on a du multithreading, et il ne fait pas parti de la norme
%% POSIX donc son exécution peut varier d'une machine à une autre. On trouve
%% ensuite les API noyau aussi appelées modules d'instrumentations telles que:
%% utrace, kprobes, uprobes{\color{red}citation}. Utrace fait la même chose que
%% ptrace mais en mode noyau. Cela permet d'éviter les nombreux changements de
%% contexte pour gérer l'appel système et que le noyau l'exécute {\color{green}
%%   clair?}. De plus il gère les événements de thread et non de processus comme
%% ptrace ce qui évite le problème de gestion du multhreading. Kprobes quant à lui
%% permet à un utilisateur d’insérer dynamiquement des points d'arrêts à des
%% endroits spécifiques du noyau, dans notre cas ce serait le code des appels
%% systèmes. Ainsi l’utilisateur peut fournir un handler particulier à exécuter
%% avant ou après l’instruction marquée. Quand un point d'arrêt est touché kprobe
%% prend la main et exécute le bon handler. Uprobes fait la même chose que kprobes
%% mais pour le code d'applications et pas le code du noyau. Ainsi pour chaque
%% point d'arrêt géré par uprobes on doit créer un module noyau qui contient le
%% handler à exécuter quand le point d'arrêt est atteint. Uprobes utilise utrace
%% pour savoir quand on atteint ou non un point d'arrêt. Les deux avantages des API
%% noyau est qu'elles sont rapides et qu'elles ont accès à toutes les ressources
%% sans aucune restriction, mais ce dernier point représente aussi leur plus gros
%% défaut de par sa dangerosité. De plus, dans notre cas il ne semble pas judicieux
%% de modifier le code du noyau ou de faire de la programmation noyau via des
%% modules dont il faudra également gérer le bon chargement. Malgrè ses défauts
%% c'est donc l'appel système ptrace qui a été choisi. Maintenant que nous savons
%% pourquoi nous allons utiliser ptrace comme intercepteur pour les appels systèmes
%% nous allons étudier son fonctionnement.

%% Pour intercepter les éventuels appels systèmes d'une applciation nous allons
%% donc utiliser l'appel système \textbf{ptrace}{\color{red}citation}. Il permet de
%% controler l'exécution de processus mais également d'écrire et de lire
%% directement dans l'espace d'adressage d'un processus. Pour cela on créé deux
%% processus; un qui exécutera l'application et qu'on souhaite contrôler, on
%% l'appellera ``processus espionné'' et un autre qui le contrôlera appelé
%% ``processus espion''. Le processus espionné indiquera au processus espion qu'il
%% souhaite être contrôlé via un appel système ptrace. À la reception de cet appel
%% le processus espion notifiera son attachement au processus espionné via un autre
%% appel à ptrace. Il indiquera également sur quelles actions du processus espionné
%% il veut être notifié, définissant ainsi les actions bloquantes pour le processus
%% espionné. Dans notre cas, ce seront les appels systèmes que l'on considérera
%% comme point d'arrêt pour le processus espionné. Ainsi quand un des processus de
%% l'application voudra faire un appel système quelconque il sera bloqué avant de
%% l'exécuter, l'appel système ptrace sera lancé et notifiera le processus
%% espion. Ce dernier fera les modifications nécessaires dans les registres du
%% processus espionné pour conserver la virtualisation de l'environnement, puis il
%% rendra la main au processus espionné bloqué pour que l'appel système puisse
%% avoir lieu. Au retour de l'appel système le processus espionné sera de noveau
%% stoppé, un ptrace sera envoye au processus espion qui remodifiera les
%% informations nécessaires. Puis il rendra la main au processus espionné bloqué
%% qui sortira de son appel système avec un résultat exécuté sur la machine hôte et
%% un temps d'exécution et une horloge fournie par le simulateur.

%% {\color{red} Mettre un schema attachement attente pere attrape signal modification main fils as retour pere...}

%% {\color{red} \textbf{gérer cette transition}} Néanmoins il a été montré dans un
%% précédent stage que l'appel système ptrace est inefficace voir inutile en ce qui
%% concerne tous les appels systèmes temporels qu'une application voudrait
%% faire. \textit{les appels systèmes ``time'', ``clock\_gettime'',
%%   ``gettimeofday'' avec \textbf{ptrace} ne sont pas possibles, d'où l'alliance
%%   avec LD\_PRELOAD} Par exemple lors d'un gettimeofday l'appel système n'est pas
%% lancé on répond directement au niveau de la bibliothèque ainsi on n'arrive même
%% pas au niveau de l'appel système, donc ptrace ne fait rien.  Problème
%% portabilité {\color{red} \textbf{gérer cette transition}}
\paragraph{uprobe}
Non module noyau

\paragraph{seqcomp/bpf}
Read only

\newpage
\section{État de l'art}
\label{section:sota}

\textbf{INTRO TODO}

\newpage
\section{Simterpose: la médiation}
\label{section:simterpose}

Dans la section précédente, nous avons présenté différents outils permettant de
faire de la virtualisation légère. Malheureusement, ils ne pouvaient pas être
utilisés dans le cadre de notre projet. Nous allons donc voir pourquoi et quel
est l'émulateur qui a été choisi pour permettre d'exécuter des applications
distribuées au dessus de SimGrid. Nous étudierons le fonctionnement interne de
ce dernier, notamment les outils présentés en section \ref{section:interception}
qu'il utilise.

%\section{Évaluation}
\label{section:evaluation}

intro

\newpage
\section{Conclusion}
\label{section:ccl}

Dans ce rapport, nous avons présenté notre objectif ainsi qu'un état de l'art
des différentes approches et outils existants. Pour pouvoir exécuter n'importe
quel type d'applications distribuées dans des conditions environnementales qui
permettront de les étudier, la meilleure solution semble être la virtualisation
par interception. Les projets existants ne permettant pas de résoudre les quatre
problèmes engendrés par cette virtualisation (gestion du temps, des threads, des
communications réseaux et le DNS), un nouveau projet a été lancé.

L'émulateur Simterpose dévelopé au LORIA permet d'exécuter et de tester des
applications distribuées réelles sans disposer de leur code sources dans un
environnement distribué virtuel. Il se base sur la plateforme de simulation
Simgrid pour mettre en place l'environnement d'exécution dans lequel
l'application pensera s'exécuter. Pour maintenir la virtualisation, les actions
des applications sont interceptées et modifiées pour ensuite être exécutées. On utilise SimGid pour calculer la réponse de l'environnement virtuel aux
différentes actions. La solution proposée intercepte les actions à deux niveaux
différents, appels systèmes et des bibliothèques, afin de ne pas en oublier. Simterpose permet également d'injecter diverses fautes dans la
simulation pour avoir une virtualisation plus réaliste.

Néanmoins, il reste à ajouter certaines fonctionalités (DNS, fonctions de temps)
afin de pouvoir vérifier qu'il est possible de mettre en place une
virtualisation par inteception qui apporte une réponse à notre
problématique. Nous projettons par la suite d'évaluer les performances de notre
projet, afin de le modifier pour le rendre plus efficace et de maximiser la
taille des expériences.



\newpage
\label{fin}

% Bib
\bibliographystyle{plainnat}
\bibliography{src/7.Biblio} % The file containing the bibliography

\end{document}
