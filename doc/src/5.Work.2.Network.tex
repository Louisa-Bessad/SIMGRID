\subsection{Réseau de communications}
Pour implémenter le réseau de communications, tel que présenté en section \ref{subsubsection:nework}, nous avons écrit pour chaque appel sysèteme réseaux une version de l'appel utilisant la \textit{médiation par traduction d'adresses} et une utilisant la \textit{full mediation}.

Dans la version de l'appel système utilisant la \textit{médiation par traduction d'adresses}, nous allons récupérer via \texttt{ptrace} les valeurs contenues dans les registres. Ensuite, selon le type d'appel système réseau dont il s'agit on effectue différentes actions sur les paramètres de l'appel. Pour les appels qui concernent l'ouverture et la fermeture de connexion (\texttt{bind}, \texttt{connect}, \texttt{listen}, \texttt{accept}, \texttt{shutdown}, \texttt{close}), ainsi que la création de socket on créé dans la table de correspondance de nouvelles entrées pour permettre de maintenir la virtualisation en traduisant les coupples <IP, port>virtuel en <IP, port>réel si il n'existent pas déjà. Pour les appels concernant les échanges de messages, afin de maintenir notre virtualisation, on cherche les valeurs passées en paramètres de l'appel qui caractérisent le réseau (port, adresse, numéro de socket...). Puis, on récupère en utilisant les tables de traductions <IP, port>virtuel / <IP, port>réel, les valeur réelles correspondantes qui nous intéressent. Pour finir, quelque soit l'appel système, on va écrire les valeurs traduites dans les registres de l'appel système grâce à notre intercepteur \texttt{ptrace} et on laisse l'appel s'exécuter. A la sortie, on refait la même chose à la seule différence qu'on traduit le couple  <IP, port>réel en <IP, port>virtuel.

Pour la version de l'appel système utilisant la \textit{full mediation}, on récupère comme précédement les paramètres de l'appel système contenus dans les registres. On neutralise l'appel système via \texttt{ptrace} pour empêcher son exécution quand on rendra la main à l'application. Puis, selon le type d'appel système réseau dont il s'agit on effectue différentes actions. Pour les appels qui concernent l'ouverture et la fermeture de connexion, ainsi que la création de socket on ne fait rien de particulier puisque dans ce type de médiation aucune socket n'est créée et connectée. Pour les appels systèmes qui correspondent à l'échange de messages on va créer une tâche SimGrid qui va permettre d'écrire ou de lire les données à envoyer ou recevoir. Puis, pour tous les appels systèmes, on va écrire dans les registres de l'appel avec \texttt{ptrace}, la valeur de retour et tout ce qui doit être modifié dans les registres en se basant sur ce que fait réellement l'appel quand il s'exécute réellement. Par exemple, quand on reçoit des données, les données lues dans la tâche SimGrid vont être écrites dans le bufer d'écriture dont l'adresse passée en paramètre est contenue dans un registre.
