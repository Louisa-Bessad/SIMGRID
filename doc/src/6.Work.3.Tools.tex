\subsection{Améliorations apportées à Simterpose}

Au début de ce stage, Simterpose ne pouvait être exécuté que sur des architectures 64bits. Considérant qu'il est important qu'un tel programme puisse s'exécuter sur tous les types d'architecture nous avons voulu résoudre ce problème afin qu'il puisse s'exécuter sur des machines 32bits. Il existe plusieurs différences entre les architectures 32bits et 64bits. Celles qui nous intéressent sont celles qui sont susceptibles d'affecter Simterpose. La première différence est que les registres ne portent pas les mêmes noms selon le type d'architecture qu'on utilise, comme le montre le tableau \ref{register}. Or, lorsqu'on intercepte des appels systèmes il faut pouvoir récupérer les valeurs contenues dans les registres à l'entrée de l'appel système puis écrire dans ces mêmes registres à la sortie. Afin d'exécuter Simterpose sur ces deux architectures il faut donc avoir une version du code pour chacune. Chaque version utilisant les bons noms de registres pour récupérer les valeurs qu'ils contiennent lors de l'interception d'appels systèmes via \texttt{ptrace}. Pour trouver sur quelle type de plateforme Simterpose est en train de s'exécuter on teste la valeur maximale que peut avoir un pointeur en mémoire avec la \texttt{macro} \texttt{UINTPTR\_MAX}. Pour avoir une architecture 64bits UINTPTR\_MAX doit valoir 0xffffffffffffffff et pour être en 32bits elle doit valoir 0xffffffff. La seconde différences est que certaines \texttt{macro} et appels systèmes n'existent pas sur des architectures 64bits et inversement. C'est la cas de deux appels systèmes particulièrement importants car ils touchent au réseau: \texttt{send} et \texttt{recv}. Ces deux appels ne sont définis que pour des architectures 32bits, sur des architectures 64bits quand on les utilise ils sont remplacés par les appels sytèmes \texttt{sendto} et \texttt{recvfrom}.

\begin{table}[H]
\centering
\begin{tabular}{lcccccccc}
\cline{2-9}
\multicolumn{1}{l|}{}              & \multicolumn{1}{c|}{{\bf \begin{tabular}[c]{@{}c@{}}Numéro de\\ l'appel système\end{tabular}}} & \multicolumn{1}{c|}{{\bf \begin{tabular}[c]{@{}c@{}}Valeur \\ de retour\end{tabular}}} & \multicolumn{1}{c|}{{\bf arg0}} & \multicolumn{1}{c|}{{\bf arg1}} & \multicolumn{1}{c|}{{\bf arg2}} & \multicolumn{1}{c|}{{\bf arg3}} & \multicolumn{1}{c|}{{\bf arg4}} & \multicolumn{1}{c|}{{\bf arg5}} \\ \hline
\multicolumn{1}{|c|}{{\it 32bits}} & \multicolumn{1}{c|}{orig\_eax}                                                                 & \multicolumn{1}{c|}{eax}                                                               & \multicolumn{1}{c|}{edi}        & \multicolumn{1}{c|}{esi}        & \multicolumn{1}{c|}{edx}        & \multicolumn{1}{c|}{r10d}       & \multicolumn{1}{c|}{r8d}        & \multicolumn{1}{c|}{r9d}        \\ \hline
\multicolumn{1}{|c|}{{\it 64bits}} & \multicolumn{1}{c|}{orig\_rax}                                                                 & \multicolumn{1}{c|}{rax}                                                               & \multicolumn{1}{c|}{rdi}        & \multicolumn{1}{c|}{rsi}        & \multicolumn{1}{c|}{rdx}        & \multicolumn{1}{c|}{r10}        & \multicolumn{1}{c|}{r8}         & \multicolumn{1}{c|}{r9}         \\ \hline
                                   & \multicolumn{1}{l}{}                                                                           & \multicolumn{1}{l}{}                                                                   & \multicolumn{1}{l}{}            & \multicolumn{1}{l}{}            & \multicolumn{1}{l}{}            & \multicolumn{1}{l}{}            & \multicolumn{1}{l}{}            & \multicolumn{1}{l}{}           
\end{tabular}
\caption{Nom des différents registres d'un appel système selon le type d'architecture}
\label{register}
\end{table}

La seconde amélioration est une mise à niveau de Simterpose pour qu'il puisse utiliser les dernières version de SimGrid. Pour cela, il a fallu remplacer d'anciennes fonctions et variables encore utilisées dans le code de Simterpose par les nouvelles utilisées dans SimGrid. Cela a également permi de mettre à jour un problème dans SimGrid dû à l'accès d'un pointeur dont on ne vérifiait pas qu'il n'était pas nul. Au début de mon stage, Simterpose utilisait une version de SimGrid datant de 2011, maintenant il utilise la version f42adf1 de git sortie le 16 Août 2015.

\vspace{0.5cm}
Troisièmememnt, nous souhaitons pouvoir utiliser un autre débogueur en plus de \texttt{gdb}. Nous avons choisi Valgrind pour les nombreux modules qu'il fourni,cité en section \ref{subsubsection:valgrind}, notamment \texttt{memcheck} qui est pour nous le plus intéressant. Ce dernier traque les fuites mémoires et résume en fin d'exécution tout ce qui a été réservé, libérée et perdu en mémoire. Pour permettre son utilisation avec Simterpose nous avons dû implémenter l'appel système \texttt{fcntl}. Valgrind utilise cet appel système pour accéder à l'exécution de Simterpose et ainsi chercher les fuites mémoires. L'implémentation d'un handler pour cet appel n'était pas prévu à la création de Simterpose puisque nous voulons juste intercepter les appels systèmes réseaux, temporels et gérer les processus et leurs threads pour maintenir notre environnement virtuel et aucun débogueur autre que \texttt{gdb} ne souhaitait être utilisé à ce moment-là.

 \vspace{0.5cm}
Pour finir, le but de Simterpose étant d'exécuter des applications distribuées large échelle, nous devons pouvoir exécuter des applciations de Torrent. De fait, lorsqu'on exécute des applications de ce type, le système de fichier de la machine hôte est en permanence utilisé. Simterpose dispose en parallèle de son propre système de fihier. Pour chaque socket ou fichier il dispose d'un descripteur avec des compteurs de références, les processus qui les référencent, les verrous qui peuvent être posés... Nous devons donc nous aussi maintenir à jour notre système de fichier dans le cas où nous lancerions ce genre d'applications. Ainsi, nous devons maintenant intercepter les appels systèmes touchant aux fichiers en plus de ceux affectant le réseau en utilisant toujours \texttt{ptrace}. Lorsqu'on les intercepte il faut récupérer les modifications qui seront effectués sur le système de fichiers réel une fois qu'on aura laissé passer l'appel système et les appliquer au système de fichier propre à Simterpose. Actuellement, Simterpose gére les appels systèmes: \texttt{open},  \texttt{close}, \texttt{creat}, \texttt{dup}, \texttt{dup2}, \texttt{poll}, \texttt{fcntl}, \texttt{lseek}, \texttt{read}, \texttt{write}. Il est possible que Simterpose soit amener à gerer d'autre appels systèmes de ce type. Néanmoins, pour le moment nous avons choisi d'implémenter les appels qui nous semblent essentiels pour bien gérer notre système de fichiers.
