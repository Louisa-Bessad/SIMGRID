\subsection{Conclusion}

Dans cette section, nous avons présenté différents outils permettant de
faire de l'interception et de la médiation d'actions d'applications, résumées
dans le tableau \ref{TAB_COMP}. Dans le cas d'émulateur ne souhaitant pas modifier
le code source d'une application, les outils présentés en \ref{subsection:source}
sont inutiles.

\begin{table}[h]
\centering
\begin{tabular}{c|c|c|c|c|c|}
\cline{2-6}
 & \texttt{ptrace} & Uprobes & seccomp/BPF & LD\_PRELOAD & Valgrind \\ \hline
\multicolumn{1}{|c|}{\begin{tabular}[c]{@{}c@{}}Niveau \\ d'interception\end{tabular}} & \begin{tabular}[c]{@{}c@{}}Appel\\ Système\end{tabular} & \begin{tabular}[c]{@{}c@{}}Appel\\ Système\end{tabular} & \begin{tabular}[c]{@{}c@{}}Appel\\ Système\end{tabular} & Bibliothèque & Binaire \\ \hline
\multicolumn{1}{|c|}{Coût} & Moyen & Faible & Moyen & Faible & Important \\ \hline
\multicolumn{1}{|c|}{\begin{tabular}[c]{@{}c@{}}Mise en\\ \oe uvre\end{tabular}} & \begin{tabular}[c]{@{}c@{}}Assez\\ complexe\end{tabular} & \begin{tabular}[c]{@{}c@{}}Assez\\ complexe\end{tabular} & \begin{tabular}[c]{@{}c@{}}Assez \\ complexe\end{tabular} & Simple & Complexe \\ \hline
\multicolumn{1}{|c|}{Utilisé pour} & \begin{tabular}[c]{@{}c@{}}- Thread \\ (incomplet)\\ - Echanges \\ réseau\end{tabular} & \begin{tabular}[c]{@{}c@{}}- Thread \\ (incomplet)\\ - Echanges\\ réseau\end{tabular} & \begin{tabular}[c]{@{}c@{}}- Thread \\ (incomplet)\\ - Echanges\\ réseau\end{tabular} & \begin{tabular}[c]{@{}c@{}}- Thread \\(incomplet)\\ - Temps\\ - DNS\end{tabular} & \begin{tabular}[c]{@{}c@{}}- Thread\\ - Temps\\ - Echanges\\ réseau\\ - DNS\end{tabular} \\ \hline
\end{tabular}
\caption[Comparaison des différentes solutions d'interception]{Comparaison des différentes solutions d'interception entre une application et le noyau.}
\label{TAB_COMP}
\end{table}

 De par le surcoût d'utilisation de Valgrind, cette solution est à écarter dans
le cas d'applications distribuées large échelle s'exécutant dans un
environnement distribué. Face aux trois solutions potentielles d'interception au
niveau des appels systèmes, nous avons fait le choix arbitraire
d'utiliser \texttt{ptrace}. De plus, nous pouvons voir qu'il y a une certaine
complémentarité entre l'appel système \texttt{ptrace} et la variable
d'environnement \texttt{LD\_PRELOAD}. En effet, \texttt{LD\_PRELOAD} résout les
lacunes de \texttt{ptrace} concernant les fonctions de temps et le
multithreading. A l'inverse, \texttt{ptrace} permet d'intercepter les appels
systèmes que l'on ne peut pas gérer avec \texttt{LD\_PRELOAD}.
