\subsubsection{Action sur le binaire}
%%Valgrind (perf pourrie)

Pour agir sur le binaire d'une application, c'est l'outil d'instrumentation
d'analyse dynamique Valgrind \citep{Valgrind, Valgrindweb} que nous allons
étudier. D'autres outils suivent la même approche générale, tels que DynInst, un outil pour la modification à la volée de binaire utilisé notamment en HPC \citep{Dyn}.

À l'origine, Valgrind était utilisé pour le débogage mémoire, puis il a évolué
pour devenir l'instrument de base à la création d'outils d'analyse dynamique de
code, tels que la mise en évidence de fuites mémoires ou le
profilage\footnote{Méthode visant à analyser le code d'une application pour
connaître la liste des fonctions appelées et le temps passé dans chacune
d'elles.}. Valgrind fonctionne à la manière d'une machine virtuelle faisant de la
compilation à volée\footnote{Technique basée sur la compilation de byte-code et
la compilation dynamique. Elle vise à améliorer la performance de systèmes
bytecode-compilés par la traduction de bytecode en code machine natif au moment
de l'exécution.}. Ainsi, ce n'est pas le code initial du programme qu'il envoie
au processeur de la machine hôte. Il traduit d'abord le code dans une forme
simple appelée ``Représentation Intermédiaire''. Ensuite, un des outils
d'analyse dynamique de Valgrind peut être utilisé pour faire des transformations
sur cette ``Représentation Intermédiaire''. Pour finir, Valgrind traduit la
``Représentation Intermédiaire'' en langage machine et c'est ce code que le
processeur de la machine hôte va exécuter. De plus, grâce à la compilation
dynamique, Valgrind peut recompiler certaines parties du code d'un programme
durant son exécution et donc ajouter de nouvelles fonctions au code de
l'application.

Dans notre cas, on peut utiliser Valgrind pour mesurer le temps passé à faire un
calcul. Ce dernier étant ensuite envoyé au simulateur pour calculer le temps de
réponse dans l'environnement simulé nécessaire à l'émulateur. On peut
également l'utiliser pour réécrire à la volée le code des fonctions que
l'émulateur doit modifier pour maintenir la virtualisation. Pour faire cela, il
faut créer un ``wrapper'' pour chaque fonction qui nous intéresse. Un wrapper
est une fonction de type identique à celle que l'on souhaite intercepter, mais
ayant un nom différent (généré par les \texttt{macro} de Valgrind) pour la
différencier de l'originale. Pour générer le nom du wrapper avec
les \texttt{macro} de Valgrind on doit préciser la bibliothèque qui contient la
fonction originale. Cela implique donc de connaître pour chaque fonction à
intercepter le nom de la librairie qui l'implémente. Cette solution est donc
assez contraignante et ses performances sont assez médiocres d'après l'étude
faite par M. Guthmuller lors de son stage \citep{MARION:Interception}: facteur
de 7.5 pour le temps d'exécution d'une application avec cet outil. Cette perte
de performance est due à la compilation faite en deux phases ainsi qu'au temps
nécessaire aux outils de Valgrind pour modifier ou rajouter du code à
l'existant. Cela pourrait être acceptable, si Valgrind faisait de la traduction
dynamique lors de la seconde phase de sa compilation, nous permettant ainsi
d'avoir du code exécutable sur un autre type de processeur que celui de l'hôte,
mais ce n'est pas le cas. Néanmoins, même si on résout le problème de
performance, la mise en \oe uvre de cette approche restera difficile.
