\subsection{MicroGrid}
\label{subsection:MicroGrid}

Les grilles sont des environnements très hétérogènes, que ce soit en terme de configuration, de performance ou de fiabilité. Les logiciels qui les utilisent doivent donc avoir une certaine flexibilité pour pouvoir s'adapter aux différents environnements et ressources disponibles. Il existe des applications qui permettent de vérifier que l'accès aux ressources d'une grille est équitable et sécurisé. Néanmois, il n'y a pas d'outil pour étudier le comportement d'applications développées pour utiliser de tels environnements et exploiter leurs ressources. On ne peut donc pas connaître la robustesse et l'efficacité de ces applications, ainsi que l'impact qu'elles ont sur la stabilité de la grille. 

MicroGrid \citep{MICROGRID_INIT, MICROGRID_CASANOVA} est un émulateur par interception créé pour résoudre ce problème. Il fournit via l'émulation une grille virtuelle large échelle permettant d'exécuter des applications, sans qu'elles soient modifiées, selon les ressources disponibles sur la grille et les différentes topologies réseaux qui peuvent être utilisées par l'application. De plus, la virtualisation permet de gérer toutes les grilles, que leurs ressources soient homogènes ou hétérogènes. Grâce au simulateur, %% utilisé dans la virtualisation par interception, 
MicroGrid peut émuler des machines plus puissantes et donc tester les applications sur des grilles qui n'existent pas encore. En définissant les ressources et le réseau à émuler on peut prédire les performances d'applications développées pour la grille. Le but n'est pas d'avoir des prédictions parfaites mais de parvenir au moins à des estimations de performances qui soient fiables dans le cas de l'exécution d'une application utilisant une topologie inexistante. L'émulateur virtualise l'environnement et le simulateur modélise les ressources de la grille (calcul, mémoire et réseau) pour calculer le temps dans l'environnement virtuel.

Pour maintenir la grille virtuelle, l'émulateur doit gérer deux types de ressources: celles pour le réseau et celles pour le calcul. Dans le premier cas, l'émulateur intercepte toutes les actions faites par l'application qui vont utiliser les ressources simulées (\texttt{gethostname}, \texttt{bind}, \texttt{send}, \texttt{receive}). L'interception se fait au niveau des bibliothèques, via \texttt{LD\_PRELOAD}. Ces appels sont ensuite transmis à l'émulateur de paquets réseau utilisé par MicroGrid, MaSSF, pour gérer les communications qui vont réellement transiter sur le réseaux. MaSSF est capable de gérer de très nombreux protocoles réseaux.  Pour ce qui est des ressources de calcul, MicroGrid utilise un contrôleur de CPU qui virtualise les ressources du CPU et gère les processus des machines virtuelles via des \texttt{SIGSTOP} et \texttt{SIGCONT}. Ce contrôleur agit à trois niveaux \textit{i)} il intercepte les appels de fonctions pour créer ou tuer des processus, toujours via \texttt{LD\_PRELOAD}, pour maintenir une table des processus virtuels à jour \textit{ii)} périodiquement il mesure le temps d'utilisation du CPU de chaque processus contenu dans sa table \textit{iii)} il ordonne les processus de chaque hôte virtuel qui sont contenus dans sa table en fonction des mesures qu'il fait.

Dans le cas de grilles hétérogènes il faut obtenir une simulation équilibrée. Autrement dit une simulation qui ne crée pas de délais à cause des temps de réponses qui diffèrent entre les plateformes. Pour cela, il faut mettre en place un mécanisme de coordination globale. MicroGrid utilise un temps virtuel pour coordonner l'écoulement du temps sur les différentes plateformes.

L'avantage de la solution proposée par MicroGrid est qu'elle peut utiliser de nombreux protocoles réseaux complexes pour émuler un réseau réaliste et qu'elle permet d'émuler des machines avec des vitesses très variables grace au contrôleur qui gère la simulation.

Le gros problème de MicroGrid est que le temps n'est pas intercepté mais émulé par dilatation \citep{MICROGRID_lee2014integrated}. Il est très compliqué de trouver le bon facteur de dilatation et de conserver la synchronisation qu'il engendre. De plus, le réseau peut ne pas gérer parfaitement cette dilatation et prendre du retard sur le CPU.

Aujourd'hui il n'existe plus de version maintenue de MicroGrid mais l'approche utilisée par ce projet a été réutilisée dans de nombreux projet notamment Timekeeper\footnote{Permet à chaque conteneur LXC d'avoir sa propre horloge virtuelle et de pouvoir faire des pauses ou des sauts dans le temps. Pour cela la fonction \texttt{gettimeofday} a été réimplémentée afin qu'elle renvoie un temps virtuel.} \citep{MICROGRID_lamps2014timekeeper} et Integrated simulation and emulation using adaptative time dilation\footnote{Ce projet  dilate le temps pour garder une certaine synchronisation entre applications et simulateur. Quand le simulateur est surchargé il prend du retard sur le temps réel et introduit des délais quand il répond à l'émulateur. Ici la fonction \texttt{gettimeofday}() a été remplacée par une \texttt{fonction get\_virtual\_time} et l'émulation se fait par dégradation avec un émulateur type KVM.} \citep{MICROGRID_lee2014integrated}.
