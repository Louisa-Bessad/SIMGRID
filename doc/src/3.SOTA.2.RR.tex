\subsection{RR}
 %% pourquoi (tester firefox), comment (ordre des threads -> perf API dans le CPU)

\citep{RR}
RR outil de debug utilisant les cmd gdb utilisé au depart pour debug firefox
les appli sont non deterministes execution tjs diff et bug peut arriver apres x execution c difficile de debug de maniere traditionnelle => temps a debugger de meme pr les bug transitoire (pas reproductible) pr les groses applis n'importe quel bug est important et doit être résolu d'ou rr
ENREGISTRE EXEC NONDETERMINISTE ET DEBUG DETERMINISTE
on enregistre une execution qui a échoué et on fait du debug sur l'enregistrement en le rejouant aussi souvent qu'on veut car la même éxécution est relancé.

RR = deterministe debug = sv nondeterministic failure une fois et la debug de facon deterministe poru tjs

on enregistr l'application du cout c toute l'execution et l'echec qui sont sv sur le disque.  du coup on peut debug la panne
ainsi on debug une trace de façon deterministe pas une execution non deterministe puisque a chaque execution rejouée toutes les ressources et actions sont les mêmes (espace d'adressage, contenu des registres, AS)

RR solve the problem en effectuant ke debug en 2 phases: recording (sv execution de appli) + debug deterministe de la sv en utilisant gdb pour controler le rejeu autant de fois que desiré

capture tous les evenements non deterministes
en rejouant ds bon ordre rr garantit que chaque session debu est deterministe: adresse, registre... ne changent pas
du coup on meme rejouer les fautes faites par des outils de fuzzing ou d'injection de fautes et les debbuguer

advantage:
diminue le cout de debbug et permet de résoudre de nv bug
fonctionne avec bcp d'appli puisque fonctionne firefox
faible sur-cout de temps puisque veut que rr remplace gdb doit pouvoir resoudre pb aussi vite :D overhead depend du type de test faits

neg:
emule une machine mono coeur donc prog parallele ralentis
ne peut pas enregistrer mem dont proc partage => thread pb
tout les AS pas encore implem donc risque de voir apparaître selon appliaction et AS que son processus fait
