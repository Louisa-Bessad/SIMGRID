\section{Introduction}

%% intro/objectif: virtualisation légère d'applications distribuées (tester des applications distribuées réelles: test regression et performance, légère car on veut tester des centaines d'instances)

%% Applications: stockage distribué (CEPH, TAHOE/LAFS) et RT event processing (Storm)

Dans le cadre de ce stage, nous allons nous intéresser aux applications
distribuées. Il s'agit d'applications dont une partie ou la totalité des
ressources n'est pas localisée sur la machine où l'application s'exécute, mais
sur plusieurs machines distinctes. Ces dernières communiquent entre elles via le
réseau pour s'échanger les données nécessaires à l'exécution de
l'application. Les applications distribuées ont de nombreux avantages: elles
permettent notamment d'augmenter la disponibilité des données en se les
échangeant, comme les applications Torrent (BitTorrent ...). Grâce au
projet BOINC\footnote{\url{https://boinc.berkeley.edu/}} par exemple, on peut
partager la puissance de calcul inutilisée de sa machine. On peut également
penser aux applications de stockage de données
LAFS\footnote{\url{https://tahoe-lafs.org/trac/tahoe-lafs}} et
CEPH\footnote{\url{http://ceph.com/}}. La première apporte un stockage robuste
qui préserve les données même si un site est physiquement détruit. La seconde
souhaite fournir performance, fiabilité et scalabilité. Depuis longtemps, la popularité de ces applications distribuées ne cesse de
croître. Elles deviennent de plus en plus complexes avec des contraintes et des
exigences de plus en plus fortes, en particulier au niveau des performances et
de l'hétérogénéité des plateformes et des ressources utilisées. Il est donc de
plus en plus difficile de créer de telles applications (absence d'horloge
commune et mémoire centrale, deadlock, race condition, famine) mais aussi de les
tester. En effet, l'évaluation d'applications distribuées a évoluée mais cela représente un effort permanent.
\newline

Actuellement, il existe trois façons principales de tester le comportement
d'applications distribuées \citep{gustedt2009experimental}: l'exécution sur
plate-forme réelle, la simulation et l'émulation.

La première solution consiste à exécuter réellement l'application sur un parc de
machines et d'étudier son comportement en conditions réelles. Cela permet de la
tester sur un grand nombre d'environnements. L'outil créé et développé en partie
en France pour cela est \textbf{Grid'5000}\footnote{Infrastructure de 8000 c\oe
  urs répartis dans la France entière créée en
  2005. \\ \url{https://www.grid5000.fr/mediawiki/index.php/Grid5000:Home}} \citep{GRID5000},
un autre outil développé à l'échelle mondiale est
\textbf{PlanetLab} \footnote{Créée en 2002, cette infrastructure de test compte
  aujourd'hui 1340 c\oe urs. \\ \url{http://www.planet-lab.org}}. Néanmoins,
afin de mettre en \oe uvre ces solutions complexes, il faut disposer des
infrastructures nécessaires pour effectuer les tests. Il faut également écrire
une application complète capable de gérer toutes ces ressources
disponibles. Enfin, du fait du partage des différentes plateformes entre
plusieurs utilisateurs, les expériences sont difficilement reproductibles.

La seconde solution consiste à faire de la simulation: on modélise ce que l'on
souhaite étudier (application et environnement) via un programme appelé
simulateur. Dans ce cas, pour pouvoir tester des applications distribuées, on doit d'abord abstraire l'application ainsi que l'environnement
d'exécution. Pour cela, on identifie les propriétés de l'application et de son
environnement puis on les transforme à l'aide de modèles mathématiques. Ainsi,
on va exécuter dans le simulateur le modèle de l'application et non
l'application réelle, dans un environnement également modélisé. Cette solution
est donc facilement reproductible, plus simple à mette en \oe uvre, et permet de
prédire l'évolution du système étudié grâce à l'utilisation de modèles
mathématiques. De nos jours, les simulateurs tels que
\textbf{SimGrid} \citep{casanova:hal-01017319} peuvent simuler des
applications distribuées mettant à contribution des milliers de
n\oe uds. Néanmoins, avec la simulation on ne peut valider qu'un modèle et non
l'application elle-même.

La troisième solution consiste à faire de l'émulation: on exécute réellement
l'application mais dans un environnement virtualisé grâce à un logiciel,
l'émulateur. Ce dernier joue le rôle d'intercepteur pour virtualiser
l'environnement d'exécution.
%On fera ainsi croire à l'application qu'elle s'exécute sur une machine autre que l'hôte.
Cette solution représente un intermédiaire entre la simulation et l'exécution
sur plateforme réelle visant à résoudre les limitations de ces deux
solutions. En effet, les actions de l'application sont réellement exécutées sur
la machine hôte (la machine réelle sur laquelle s'exécute l'émulation) et grâce
à la virtualisation, l'application pense être dans un environnement différent de
la machine réelle. De plus, cela évite d'avoir deux versions de l'application en
terme de code: une pour la simulation et une pour la production. L'émulation
peut-être faite \textit{off-line} (on sauvegarde les actions de l'application
sur disque et on les rejoue plus tard dans un simulateur) ou \textit{on-line}
(on bloque l'application le temps que le temps de réponse de la plateforme
virtualisée soit calculé).

%% Il existe deux types d'émulation pour les applications distribuées; la
%% virtualisation standard et la légère. On parle de virtualisation légère quand on
%% souhaite tester des applications sur une centaine d'instances.

Au cours de ce stage, nous allons nous intéresser plus particulièrement à
l'exécution d'applications distribuées arbitraires au dessus de la plateforme de
simulation SimGrid. Pour cela, nous allons présenter en section
\ref{section:emulation} les différents types de virtualisation qui
existent. Nous verrons en section \ref{section:tools} les outils qui permettent
de mettre en place la virtualisation que nous aurons choisie pour ce
projet. Puis en section \ref{section:sota} nous présenterons les projets
permettant de faire de l'émulation pour tester des applications dans un
environnement distribué. Ensuite, nous expliquerons en section
\ref{section:simterpose}, pourquoi dans le cadre du projet Simterpose c'est la
virtualisation légère par interception qui a été choisie et comment elle
fonctionne. En section \ref{section:work}, nous expliquerons le travail qui a
été effectué durant ce stage. Puis, nous présenterons nos expériences et les
résultats obtenus en section \ref{section:evaluation}. Pour finir, nous
conclurons avec les sections \ref{section:future_work} et \ref{section:ccl}.
