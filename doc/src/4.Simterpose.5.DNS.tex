\subsubsection{DNS}
%% libcalls (ne rien rater), config fake (system wide), intercept 53 ( plus dur que nécessaire, port dns autre ou pas)

Dans le cas d'utilisation du protocole DNS, on peut vouloir modifier le
comportement de l'application afin qu'elle utilise d'autres serveurs que ceux
utilisés par défaut ou aucuns afin que la résolution soit entièrement géré par
Simterpose. Pour gérer cela plusieurs solutions sont envisageables.

Une première solution serait de faire de l'interception de communications au
niveau du port 53, utilisé par défaut dans DNS. Néanmoins, cela est assez
complexe à \oe uvre car il faut pour chaque communication faite par
l'application tester le port qu'elle souhaite utiliser. De plus, il n'est pas
impossible que l'utilsiateur définisse un autre port pour le protocole DNS que
celui par défaut. Cette solution n'est donc pas envisageable.

Une autre approche serait de remplacer le fichier \texttt{resolv.conf} utilisé
pour la résolution de nom. Ainsi, l'utilisateur configurerait son propre
fichier, il pourrait égalemment fournir un fichier de spécifications de
comportement en cas d'utilisation de DNS permettant à Simterpose de générer un
nouveau fichier \texttt{resolv.conf}. Néanmoins, cette solution génère une
surcharge de travail pour l'utilisateur or nous souhaitons avoir un émulateur
qui soit simple d'utilisation.

La dernière solution envisageable est de faire de l'interception d'appel de
fonctions. Dans cas, on créé une bibliothèque partagée qui réécrit les foncitons
liés à la résolution de nom que l'on inclut dans la variable d'environnemnt
\texttt{LD\_PRELOAD}. Mais on a toujours le même problème qui est de n'oublier
aucune fonction pour ne pas mettre en péril notre environnement virtuel. Pour
l'instant, c'est la solution qui a été choisie. N'ayant pas encore été mise en
place, il n'est pas exclu que nous devions trouver une autre solution pour gérer
le DNS.

\vspace{0.5cm}
