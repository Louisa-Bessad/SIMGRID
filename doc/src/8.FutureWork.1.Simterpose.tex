\subsection{Simterpose}
\subsubsection{Fonctionnalités manquantes de Simterpose}
Dans la section \ref{section:work}, nous avons présentés les deux fontcionnalités implémentées durant ce stage. Néanmoins, il reste encore deux fonctionnalités à gérer: les threads et le DNS.

L'implémentation pour la gestion des threads se fera par la double interception complémentaire de \texttt{ptrace} et \texttt{LD\_PRELOAD}, comme nous l'avions présenté dans la section \ref{section:threads}.

Pour gèrer la résolution de nom avec le DNS, trois solutions avaient été proposées en section \ref{section:DNS}, résumées dans le tableau \ref{table:DNS}. Nous continuons de penser que la meilleure solution est l'interception de fonction liées à la résolution de noms. C'est celle qui sera implémentée en premier lors de la gestion du DNS. Si cette solution ne fonctionne pas ou si les performances sont mauvaises une des deux autres solutions sera envisagée.

\begin{table}[H]
  \centering
  \resizebox{\textwidth}{!}{%
    \begin{tabular}{c|c|c|c|ll}
      \cline{2-4}
      & {\bf \begin{tabular}[c]{@{}c@{}}Interception sur \\ le port 53\\ (DNS par défaut)\end{tabular}}                                                                                                 & {\bf \begin{tabular}[c]{@{}c@{}}Remplacer le fichier\\ \texttt{resolv.conf}\end{tabular}}                             & {\bf \begin{tabular}[c]{@{}c@{}}Interception des fonctions\\ liées à la \\ résolution de noms\end{tabular}} &  &  \\ \cline{1-4}
      \multicolumn{1}{|c|}{{\it \begin{tabular}[c]{@{}c@{}}Niveau\\ d'interception\end{tabular}}} & \begin{tabular}[c]{@{}c@{}}Appel\\ Système\end{tabular}                                                                                                                                         & X                                                                                                                       & Bibliothèque                                                                                                &  &  \\ \cline{1-4}
        \multicolumn{1}{|c|}{{\it \begin{tabular}[c]{@{}c@{}}Outils\\ disponibles\end{tabular}}}    & \texttt{ptrace}                                                                                                                                                                              & X                                                                                                                       & \texttt{LD\_PRELOAD}                                                                                      &  &  \\ \cline{1-4}
        \multicolumn{1}{|c|}{{\it Coût}}                                                            & Moyen                                                                                                                                                                                           & Faible                                                                                                                  & Faible                                                                                                      &  &  \\ \cline{1-4}
        \multicolumn{1}{|c|}{{\it \begin{tabular}[c]{@{}c@{}}Mise en\\ \oe uvre\end{tabular}}}      & Complèxe                                                                                                                                                                                        & Simple                                                                                                                  & Simple                                                                                                      &  &  \\ \cline{1-4}
        \multicolumn{1}{|c|}{{\it Problèmes}}                                                       & \begin{tabular}[c]{@{}l@{}}- Teste pour chaque\\  communication le port utilisé\\ - Redéfinition possible du\\ port utilisé par le protocole DSN\end{tabular} & \begin{tabular}[c]{@{}l@{}}- Charge de travail \\ sur l'utilisateur\\ (création du fichier)\end{tabular} & - Risque d'oubli de fonctions                                                          &  &  \\ \cline{1-4}
    \end{tabular}
  }
  \caption{Solutions proposées pour gérer la résolution de noms via le protocole DSN dans Simterpose.}
  \label{table:DNS}
\end{table}

\subsubsection{Amélioration des fonctionnalités existantes de Simterpose}
En section \ref{subsubsection:fonctionnement_reseau}, nous avons présentés deux types de médiation qui ont été implémentées pendant ce stage: la \textit{full mediation} et la \textit{médiation par traduction d'adresse}. Comme nous l'avons vu en section \ref{section:work}, lorsque Simterpose utilise la \textit{full mediation} il utilise \texttt{ptrace} pour accéder aux données à envoyer et/ou recevoir.

Nous avons envisagé une autre solution appelée ``accès direct'' qui serait peut-être moins coûteuse que l'exécution de cet appel système pour ce type de requête. Ce troisième type de médiation consisterait à ouvrir l'espace d'adressage des deux processus qui sont en train de communiquer via Simterpose, accessible depuis \texttt{/proc/PID/mem}. On pourrait ainsi lire et écrire directement dans la mémoire du processus concerné. Il existe pour cela des appels systèmes qui permettent à l'hypervisor d'optimiser les communications entre VM\footnote{\url{http://knem.gforge.inria.fr/}}, une VM représentant dans notre cas un processus essayant de communiquer avec un autre sur une autre VM. Néanmoins, il n'est pas garanti que l'utilisation de ces appels soit plus efficace que l'appel à \texttt{ptrace}.
