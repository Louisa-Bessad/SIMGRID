Dans cette section, nous avons présenté différentes approches permettant de
faire de l'interception et de la médiation d'actions d'applications, résumées
dans la table \ref{TAB_COMP}. Dans le cas d'émulateur ne souhaitant pas modifier
le code source d'une application, les outils présentés en \ref{section:source}
sont inutiles. De plus, de par le surcoût d'utilisation de Valgrind, cette
solution est à écarter dans le cas d'applications distribuées large échelle
s'exécutant dans un environnement distribué.

\begin{table}[h]
\centering
\begin{tabular}{c|c|c|c|c|c|}
\cline{2-6}
 & ptrace & Uprobes & seccomp/BPF & LD\_PRELOAD & Valgrind \\ \hline
\multicolumn{1}{|c|}{\begin{tabular}[c]{@{}c@{}}Niveau \\ d'interception\end{tabular}} & \begin{tabular}[c]{@{}c@{}}Appel\\ Système\end{tabular} & \begin{tabular}[c]{@{}c@{}}Appel\\ Système\end{tabular} & \begin{tabular}[c]{@{}c@{}}Appel\\ Système\end{tabular} & Bibliothèque & Binaire \\ \hline
\multicolumn{1}{|c|}{Coût} & Moyen & Faible & Moyen & Faible & Important \\ \hline
\multicolumn{1}{|c|}{\begin{tabular}[c]{@{}c@{}}Mise en\\ oeuvre\end{tabular}} & \begin{tabular}[c]{@{}c@{}}Assez\\ complèxe\end{tabular} & \begin{tabular}[c]{@{}c@{}}Assez\\ Complexe\end{tabular} & \begin{tabular}[c]{@{}c@{}}Assez \\ complexe\end{tabular} & Simple & Complexe \\ \hline
\multicolumn{1}{|c|}{Utilisé pour} & \begin{tabular}[c]{@{}c@{}}- Thread \\ (incomplet)\\ - Echanges \\ réseau\end{tabular} & \begin{tabular}[c]{@{}c@{}}- Thread \\ (incomplet)\\ - Echanges\\ réseau\end{tabular} & \begin{tabular}[c]{@{}c@{}}- Thread \\ (incomplet)\\ - Echanges\\ réseau\end{tabular} & \begin{tabular}[c]{@{}c@{}}- Thread \\(incomplet)\\ - Temps\\ - DNS\end{tabular} & \begin{tabular}[c]{@{}c@{}}- Thread\\ - Temps\\ - Echanges\\ réseau\\ - DNS\end{tabular} \\ \hline
\end{tabular}
\caption{Comparaison des différentes solutions d'interception entre une
  application et le noyau.}
\label{TAB_COMP}
\end{table}
