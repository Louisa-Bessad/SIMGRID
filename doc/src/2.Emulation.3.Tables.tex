Dans cette section, nous avons présenté différentes approches permettant de faire de l'interception et de la médiation d'actions d'applications, résumé dans la table \ref{TAB_COMP}. Dans le cas d'émulateur ne souhaitant pas modifier le code source d'une application les outils présentés en \ref{section:source} sont inutiles. De plus, de par le surcoût d'utilisation de Valgrind, cette solution est à écarter dans le cas d'applications distribuées large échelle s'exécutant dans un environnement distribué.

\begin{table}[H]
\resizebox{\textwidth}{!}{%
\begin{tabular}{c|c|c|c|c|c|c|}
\cline{2-7} & \texttt{ptrace} & Uprobes & seccomp/BPF & \texttt{LD\_PRELOAD} &
got Poisoning & Valgrind \\ \hline
\multicolumn{1}{|c|}{\begin{tabular}[c]{@{}c@{}}Niveau
    \\ d'interception\end{tabular}}
& \begin{tabular}[c]{@{}c@{}}Appel\\ Système\end{tabular}
    & \begin{tabular}[c]{@{}c@{}}Appel\\ Système\end{tabular}
        & \begin{tabular}[c]{@{}c@{}}Appel\\ Système\end{tabular} & Bibliothèque
            & Bibliothèque & Binaire \\ \hline \multicolumn{1}{|c|}{Coût} &
            Moyen & Faible(?) & ? & Faible & ? & Important \\ \hline
            \multicolumn{1}{|c|}{Utilisation}
            & \begin{tabular}[c]{@{}c@{}}Assez\\ complèxe\end{tabular} & ? & ? &
                Simple & ? & Complexe \\ \hline
\end{tabular}
}
\caption{Comparaison des différentes solutions d'interception entre une
  application et le noyau.}
\label{TAB_COMP}
\end{table}
