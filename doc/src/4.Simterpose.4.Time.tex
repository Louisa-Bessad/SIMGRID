\subsubsection{Le temps}
 %% -> syscall (- system wide), VDSO-linker (cross process ou VDSO)

Nous souhaitons que l'écoulement du temps perçu par les applications soit un
temps virtuel, celui qui s'écoulerait dans l'environnement virtuel, pour que
l'émulation soit plus réaliste. Il y a deux types d'actions à différencier pour
gérer le temps: les appels de fonctions liées au temps et les actions faites par
l'application qui prennent du temps lors de leur exécution.

Dans le premier cas, Simterpose va intercepter les appels de fonctions
temporelles et les modifier pour renvoyer l'heure virtuelle. Ces dernières étant
assez nombreuses, il est moins coûteux, en terme charge de travail pour
le programmeur, d'intercepter via
\texttt{ptrace} les appels sytèmes temporels (\texttt{time},
\texttt{clock\_gettime}, \texttt{gettimeofday}).

Néanmoins, il a été montré dans un précédent stage
\citep{CHLOE:Emulationapplicationdistribuees} que \texttt{ptrace} est inefficace
voire inutile en ce qui concerne l'interception des appels systèmes temporels
qu'une application souhaiterait exécuter car le noyau ne les exécute pas. Cela
est dû à l'existence de la bibliothèque \textit{Virtual Dynamic Shared Object}
(\texttt{VDSO}). Cette dernière vise à minimiser les coûts dûs aux deux changements de contexte effectués lors de l'exécution d'un appel système. \texttt{VDSO} va retrouver l'heure dans un segment du processus partagé avec l'espace noyau lisible par tous les processus sans changer de mode. Il est possible de désactiver cette bibliothèque lors du boot mais cela réduit les performances, augmente le nombre de changements de contexte, et oblige l'utilisateur à modifier les paramètres de son noyau.

On va donc se placer à un autre niveau pour intercepter ces fonctions. Malgré
leur nombre, il a été décidé d'agir lors du préchargement de bibliothèque. On va
créer une bibliothèque qui surcharge les appels de fonctions liées au temps et
la placer dans la variable d'environnement \texttt{LD\_PRELOAD}. Une fois l'interception temporelle effectuée,
Simterpose interroge SimGrid pour obtenir l'heure virtuelle et la renvoie à
l'application.

Dans le second cas, lors d'une action interceptée (calcul, communications
réseau), Simterpose doit gérer l'horloge de l'application avant de lui renvoyer
le résultat de l'exécution. Pour cela, Simterpose envoie à SimGrid l'heure et la
durée d'exécution sur la machine hôte . Ce dernier calcule en fonction de ces
deux informations, le temps qui aurait été nécessaire pour une telle exécution
sur la plateforme virtuelle et l'envoie à Simterpose. Pour finir, l'émulateur
rend la main à l'application en lui envoyant l'heure virtuelle en plus du
résultat de son action.
