\subsubsection{Les threads}
 %% syscall clone + libcalls

La gestion des threads se fera à deux niveaux dans Simterpose. On fera appel à l'interception d'appels système via \texttt{ptrace} pour tout ce qui concerne la création de threads (\texttt{fork}, \texttt{clone}, \texttt{pthread\_create}). Pour le reste on utilisera un autre outil, car certains mécanismes utilisés par les threads ne passent pas par des appels systèmes pour s'exécuter, par exemple les futex\footnote{Fast User-space mutex \url{http://man7.org/linux/man-pages/man7/futex.7.html}}. On utilise donc le préchargement de bibliothèques dynamiques en complément. Comme nous l'avons vu en section \ref{paragraphe:LDPreload}, nous allons créer une librairie contenant toutes les fonctions utilisées par les threads que nous voulons intercepter (\texttt{pthread\_init}, \texttt{pthread\_join}, \texttt{pthread\_exit}, \texttt{pthread\_yield} ...). On placera ensuite la bibliothèque dans la variable d'environnement \texttt{LD\_PRELOAD} pour que nos fonctions passent avant les autres.


