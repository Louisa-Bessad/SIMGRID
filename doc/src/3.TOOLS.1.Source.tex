\subsection{Action sur le fichier source}
\label{subsection:source}
%% reimplem SMPI (trop spé) ,source to source/ pass LLVM( gcc+libc=consanguin) 
%% , Coccinelle
Le premier niveau auquel on peut se placer pour intercepter les actions est le fichier source de l'application. On peut réécrire les parties qui nous intéressent avant de compiler le code.

Un premier outil pour cela est le programme Coccinelle\footnote{Coccinelle project \url{http://coccinelle.lip6.fr/}}. Il permet de trouver et transformer automatiquement des parties spécifiques d'un code source C. Pour cela, Coccinelle fournit le langage \texttt{SmPL}\footnote{Semantic Patch Language} permettant d'écrire les patchs sur lesquels il va se baser pour transformer le code. Un patch contient une suite de règles, chacune transforme le source en ajoutant ou supprimant du code. Lors de son exécution, Coccinelle scanne le code et cherche les lignes qui remplissent les conditions des règles spécifiées dans le patch et applique les transformations correspondantes. Dans notre cas, il s'agirait de toutes les actions de communications directes ou indirectes avec le noyau susceptibles de mettre à jour l'environnement virtuel. Néanmoins, il ne faut pas oublier de définir une règle pour chacune de ces actions sinon l'interception sera contournée. De plus, il faut pouvoir accéder au fichier source pour le modifier, or cela n'est pas toujours possible.

Une seconde solution beaucoup plus spécifique est de réimplémenter totalement la bibliothèque de communications utilisée par l'application. Cette approche est par exemple utilisée par le projet SMPI\footnote{SMPI Project: Simulation d'applications MPI \\ \url{http://www.loria.fr/~quinson/Research/SMPI/}} \citep{clauss2011single}, qui réimplémente le standard MPI au dessus du simulateur SimGrid. Cette approche manque de  généricité car ce travail est à refaire pour chaque bibliothèque de communication existante. 

