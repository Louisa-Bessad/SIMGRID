\subsection{Temps}
Pour maintenir notre environnement virtuel, nous ne pouvons pas laisser les applications accéder aux horloges de la machine hôte, en utilisant pour cela la bibliothèque \texttt{VDSO}. En section \ref{subsubsection:time}, nous avions proposé d'utiliser la variable d'envieronnement \texttt{LD\_PRELOAD}, présentée en section \ref{paragraphe:LDPreload}, afin d'éviter les appels à cette bibliothèque. En bloquant ces appels l'application ne peut plus accéder au temps de la machine.

Néanmoins, pour que notre virtualisation soit parfaite il faudrait que le temps que l'applicatioin voit s'écouler soit celui qui s'écoulerait sur la machine simulée. Afin de permettre cela, j'ai créé une bibliothèque de fonction temporelle qui est ensuite placée dans la variable d'environnement \texttt{LD\_PRELOAD} pour appeler les nouvelles fonctions et non les fonctions usuelles. Pour chaque fonction de la \texttt{libc} qui fasse appel à une des horloges de la machines (\texttt{ftime}, \texttt{time}...), j'ai implémenté une nouvelle fonction. Pour chaque fonction au lieu de demander l'horloge de la machine hôte, on demande à Simterpose de nous fournir l'heure sur la machine simulée dans SimGrid. Pour demander l'heure sur la machine simulé à simGrid il faut utiliser la fonction \texttt{MSG\_get\_clock}. Or, comme le montre la Figure \ref{TODO} l'appel à la fonction de temps que nous sommes en train de réécrire se fera dans le processus fils qui exécute l'application controlée par Simterpose. Il n'y a que le processus espion de Simterpose qui puisse faire cet appel, l'application qui tourne dans le fils ``espionné'' ne peut communiquer directement avec SimGrid. Il faut donc réussir à faire intervenir le processus espion. Le seul moyen de faire cela est que la fonction temporelle que nous sommes en train de réécrire fasse un appel système. Elle ne doit pas faire n'importe quel appel système car le but de cet appel et de récuperer l'heure simulée.Il nous faut donc trouver un appel système qui lorsqu'il sera intercepté par ptrace ne fera que nous rendre l'heure et dont on n'aura jamais besoin. On ne pourrait par exemple utiliser l'appel système send pour cela car en le réimplémentant pour demander l'heure on ne pourrait plus l'utiliser pour envoyer des messages. Dans tous les noyaux, il existe des appels systèmes qui ne sont plus implémentés ou maintenus. C'est le cas de l'appel système \texttt{tuxcall}. Quand on fait appel à ce dernier le système ne fait rien, il émet juste un avertissement pour dire que l'appel n'existe plus. Nous avons donc choisi de réimplémenter cet appel. Nous allons ismplement lui demander de faire l'appel à l'horloge et d'écrire la valeur dans ses registres pour que l'application puisse les récupérer et neutraliser l'appel pour éviter l'avertissement du système qui bloquerait le programme.
\begin{itemize}
\item paragraph section précédente LD\_PRELOAD
\item pourquoi protéger temps simulation
\item nouvel AS pour gérer le temps dans Simterpose
\item deux mediation pour chaque appel pourquoi non
\end{itemize}
