\section{Introduction}
mettre sur 2 pages une présentation de l'entreprise et des activités, et sur 10 à 20 pages la présentation du contexte de la mission, les problématiques, et les solutions envisagés. Plus ce qu'on a déjà fait si on a déjà attaqué le vrai travail.
\begin{enumerate}
\item entreprise activité (relier au 3 maybe)
\item pour les tester 3 méthodes: l'exécution réelle, la simulation de l'exécution en environnement modélise et l'émulation cas où les applications s'exécutent réellement mais sur des environnements virtuels {\color{red} reprendre sujet stage à développer dans 1} 
\item Dans le cadre de ce stage nous allons nous intéresser aux applications ditribués à large échelle et comment on peut les tester et évaluer leurs perfs via une combinaison d'émulation et de simulation en utilisant SIMGRID et Simterpose qui sont deux projets européens. SIMGRID a été lancé en 1999 pour étudier des algorithmes d'ordonnancement sur des plateformes hétérogènes dans un environnement distribué et faciliter leur programmation. Il fournit les outils de base nécessaire à la simulation de ce type d'applications. Simterpose s'insère dans le projet SIMGRID afin de pouvoir étudier des applications complètes et pas uniquement leur modèle que l'on fournit habituellement en paramètre au simulateur. Le but est de faire de l'émulation en utilisant un simulateur que sera SIMGRID. Puisque nous nous intéressons aux applications distribuées notre émulateur doit pouvoir\textit{i)} exécuter un grand nombre d'instances d'une même application sur un même système afin de pouvoir debugguer, \textit{ii)} évaluer des applications ayant de nombreuses condition d'exécution (simple n\oe ud, réseau complet), \textit{iii)} collecter les informations concernant l'application pendant qu'elle s'exécute.
\item SIMGRID et Simteprose = projet europeen blablabla
\item nous ce sera la partie émulation sur laquelle nous travaillerons
\item plan maybe 2.pourquoi faire de l'émulation simulation et pas les deux autres 3. comment ça marche réellement 4.ce qu'il y a à faire et pour quand
\item on veut tester les api sans avoir accès aux codes sources.
\end{enumerate}
